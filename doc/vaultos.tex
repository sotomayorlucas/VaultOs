% =============================================================================
% VaultOS: A Capability-Secured Nanokernel with Database-Centric Architecture
% Design, Algorithms, and Implementation
% LaTeX Document — CLRS Style
% =============================================================================
\documentclass[11pt,twoside,openright]{book}

% --- Geometry ---
\usepackage[
  paperwidth=7in, paperheight=10in,
  inner=1in, outer=0.75in,
  top=0.9in, bottom=1in,
  headheight=14pt
]{geometry}

% --- Fonts and Encoding ---
\usepackage[T1]{fontenc}
\usepackage{lmodern}
\usepackage{microtype}

% --- Math ---
\usepackage{amsmath,amssymb,amsthm}

% --- Algorithms (CLRS style) ---
\usepackage[ruled,lined,linesnumbered,noend]{algorithm2e}
\SetKwProg{Fn}{Function}{:}{}
\SetKwFunction{proc}{Procedure}
\SetKwIF{If}{ElseIf}{Else}{if}{then}{else if}{else}{}
\SetKwFor{For}{for}{do}{}
\SetKwFor{While}{while}{do}{}
\SetKw{Return}{return}
\SetKw{And}{and}
\SetKw{Or}{or}
\SetKw{Not}{not}
\SetKw{True}{true}
\SetKw{False}{false}
\SetKw{Nil}{nil}
\SetKw{To}{to}
\SetKw{DownTo}{downto}
\SetKw{Error}{error}
\SetAlgoSkip{medskip}
\DontPrintSemicolon

% --- TikZ ---
\usepackage{tikz}
\usetikzlibrary{arrows.meta,shapes,positioning,calc,fit,backgrounds,chains,decorations.pathreplacing}

% --- Code listings ---
\usepackage{listings}
\lstset{
  basicstyle=\ttfamily\small,
  keywordstyle=\bfseries,
  commentstyle=\itshape\color{gray},
  stringstyle=\color{darkgray},
  numbers=left,
  numberstyle=\tiny\color{gray},
  numbersep=5pt,
  breaklines=true,
  frame=single,
  framerule=0.4pt,
  xleftmargin=1.5em,
  language=C
}

% --- Tables ---
\usepackage{booktabs,tabularx,array,multirow}

% --- Colors ---
\usepackage{xcolor}
\definecolor{chaptercolor}{RGB}{0,51,102}
\definecolor{algcolor}{RGB}{240,245,250}

% --- Headers/Footers ---
\usepackage{fancyhdr}
\pagestyle{fancy}
\fancyhf{}
\fancyhead[LE]{\small\itshape\leftmark}
\fancyhead[RO]{\small\itshape\rightmark}
\fancyfoot[C]{\thepage}
\renewcommand{\headrulewidth}{0.4pt}

% --- Chapter/Section Styling ---
\usepackage{titlesec}
\titleformat{\chapter}[display]
  {\normalfont\huge\bfseries\color{chaptercolor}}
  {\chaptertitlename\ \thechapter}{20pt}{\Huge}
\titleformat{\section}{\normalfont\Large\bfseries\color{chaptercolor}}{\thesection}{1em}{}
\titleformat{\subsection}{\normalfont\large\bfseries}{\thesubsection}{1em}{}

% --- Theorem Environments ---
\theoremstyle{plain}
\newtheorem{theorem}{Theorem}[chapter]
\newtheorem{lemma}[theorem]{Lemma}
\newtheorem{corollary}[theorem]{Corollary}
\newtheorem{proposition}[theorem]{Proposition}
\theoremstyle{definition}
\newtheorem{definition}[theorem]{Definition}
\newtheorem{property}[theorem]{Property}
\theoremstyle{remark}
\newtheorem{remark}[theorem]{Remark}
\newtheorem{observation}[theorem]{Observation}

% --- Hyperref ---
\usepackage{hyperref}
\hypersetup{
  colorlinks=true,
  linkcolor=chaptercolor,
  citecolor=chaptercolor,
  urlcolor=chaptercolor,
  pdfauthor={VaultOS Development Team},
  pdftitle={VaultOS: A Capability-Secured Nanokernel},
}

% --- Misc ---
\usepackage{enumitem}
\usepackage{caption}
\usepackage{float}

% --- Custom Commands ---
\newcommand{\BigO}{\mathcal{O}}
\newcommand{\field}[1]{\textit{#1}}
\newcommand{\const}[1]{\textsc{#1}}
\newcommand{\reg}[1]{\texttt{#1}}
\newcommand{\hex}[1]{\texttt{0x#1}}
\newcommand{\VaultOS}{\textsc{VaultOS}}
\newcommand{\xmark}{\ding{55}}

% =============================================================================
\begin{document}

% --- Title Page ---
\begin{titlepage}
\centering
\vspace*{2cm}
{\Huge\bfseries\color{chaptercolor} VaultOS\par}
\vspace{0.5cm}
{\LARGE A Capability-Secured Nanokernel with\\Database-Centric Architecture\par}
\vspace{1cm}
{\Large\itshape Design, Algorithms, and Implementation\par}
\vspace{2cm}
\begin{tikzpicture}[
  layer/.style={draw,thick,minimum width=11cm,minimum height=0.8cm,
    fill=#1,font=\sffamily\small},
]
\node[layer=blue!8]                         (hw)  {x86\_64 Hardware $\cdot$ UEFI Firmware};
\node[layer=blue!15,above=0pt of hw]        (nk)  {Nanokernel Core: Scheduler $\cdot$ Memory $\cdot$ Interrupts $\cdot$ Drivers};
\node[layer=blue!22,above=0pt of nk]        (cr)  {Cryptographic Engine: AES-128 $\cdot$ SHA-256 $\cdot$ HMAC $\cdot$ RNG};
\node[layer=blue!30,above=0pt of cr]        (cp)  {Capability Manager $\cdot$ Encrypted Database Engine};
\node[layer=blue!40,above=0pt of cp,
  font=\sffamily\small\color{white}]        (sh)  {VaultShell: Friendly CLI $\cdot$ SQL REPL $\cdot$ Script Engine};
\node[layer=blue!50,above=0pt of sh,
  font=\sffamily\small\color{white}]        (gui) {GUI Desktop: Window Manager $\cdot$ Compositor $\cdot$ Applications};
\end{tikzpicture}
\vspace{2cm}
{\large February 2026\par}
\end{titlepage}

\frontmatter
\tableofcontents

% =============================================================================
% PREFACE
% =============================================================================
\chapter*{Preface}
\addcontentsline{toc}{chapter}{Preface}

\VaultOS{} is a nanokernel operating system built on two radical premises:

\begin{enumerate}[leftmargin=2em]
\item \textbf{Everything is a database.} There is no file system, no \texttt{/proc}, no device files.
Every system resource---processes, capabilities, messages, user data---is a row in an
encrypted relational table, accessible only through a SQL-subset query language.

\item \textbf{All data is confidential.} Every database record is encrypted with AES-128-CBC
using per-table keys derived from a master secret via HMAC-SHA256. Every access
requires a cryptographically sealed capability token that proves authorization.
\end{enumerate}

\noindent
\VaultOS{} deliberately rejects POSIX. There are no file descriptors, no \texttt{fork()},
no signals, no pipes, and no user/group permission bits. Instead, the system provides
a small, formally analyzable interface: database queries mediated by unforgeable
capability tokens.

\paragraph{Conventions.}
Algorithms are presented in the pseudocode style of Cormen, Leiserson, Rivest, and
Stein~(CLRS). We use $\BigO$-notation for asymptotic analysis. Diagrams use
TikZ. Source code excerpts are in~C or x86-64 assembly.

\mainmatter

% =============================================================================
% PART I: FOUNDATIONS
% =============================================================================
\part{Foundations}

% -----------------------------------------------------------------------------
\chapter{Introduction and Design Philosophy}
\label{ch:intro}

% -----------------------------------------------------------------------------
\section{The Database-as-OS Paradigm}
\label{sec:db-paradigm}

Traditional operating systems expose heterogeneous interfaces: a file system for
persistent storage, \texttt{/proc} for process metadata, signals for asynchronous
notification, and sockets for communication. Each subsystem has its own access-control
model, its own naming scheme, and its own failure modes.

\VaultOS{} replaces this patchwork with a single abstraction: the \emph{encrypted
relational table}. Six system tables (Table~\ref{tab:system-tables}) store all kernel
state. User interaction proceeds entirely through a SQL-subset query language executed
by the \textsc{VaultShell} REPL.

\begin{table}[htb]
\centering
\caption{VaultOS system tables.}
\label{tab:system-tables}
\begin{tabular}{@{}llcp{5.5cm}@{}}
\toprule
\textbf{ID} & \textbf{Table} & \textbf{Encrypted} & \textbf{Purpose} \\
\midrule
0 & \texttt{SystemTable}     & Yes & Boot metadata (OS name, version) \\
1 & \texttt{ProcessTable}    & Yes & Active processes (PID, state, capabilities) \\
2 & \texttt{CapabilityTable} & Yes & HMAC-sealed access tokens \\
3 & \texttt{ObjectTable}     & Yes & User-defined objects (data blobs) \\
4 & \texttt{MessageTable}    & Yes & IPC message queue \\
5 & \texttt{AuditTable}      & Yes & Security audit log \\
\bottomrule
\end{tabular}
\end{table}

\begin{definition}[Database-as-OS]
An operating system in which every named resource $r$ has a canonical representation
as a tuple $\langle \mathrm{id}, \mathrm{col}_1, \ldots, \mathrm{col}_k \rangle$ in a
table $T$, and every operation on $r$ is expressed as a query
$Q \in \{\textsc{Select}, \textsc{Insert}, \textsc{Update}, \textsc{Delete}\}$ over~$T$.
\end{definition}

% -----------------------------------------------------------------------------
\section{Capability-Based Security Model}
\label{sec:cap-model}

Access control in \VaultOS{} is based on \emph{capabilities}: unforgeable tokens that
encode a subject's rights over an object.

\begin{definition}[Capability]
A capability is a tuple
\[
  c = \bigl(\mathrm{cap\_id},\; \mathrm{obj\_id},\; \mathrm{type},\;
       \mathrm{owner},\; \mathrm{rights},\; \mathrm{parent},\; \sigma\bigr)
\]
where $\sigma = \mathrm{HMAC\text{-}SHA256}(K_{\mathrm{master}},\;
\mathrm{cap\_id} \| \mathrm{obj\_id} \| \mathrm{owner} \| \mathrm{rights} \|
\mathrm{type} \| \mathrm{parent})$ is a 256-bit cryptographic seal.
\end{definition}

\noindent
Rights are encoded as a 6-bit mask:

\begin{center}
\begin{tabular}{@{}clc@{}}
\toprule
\textbf{Bit} & \textbf{Right} & \textbf{Value} \\
\midrule
0 & \const{Read}    & $2^0$ \\
1 & \const{Write}   & $2^1$ \\
2 & \const{Execute} & $2^2$ \\
3 & \const{Delete}  & $2^3$ \\
4 & \const{Grant}   & $2^4$ \\
5 & \const{Revoke}  & $2^5$ \\
\bottomrule
\end{tabular}
\end{center}

\begin{theorem}[Capability Unforgeability]
\label{thm:cap-unforge}
An adversary who does not know the master key $K_{\mathrm{master}}$ cannot produce a
valid capability $c'$ (one whose HMAC verifies) except with probability negligible in
$|K_{\mathrm{master}}|$, assuming HMAC-SHA256 is a secure PRF.
\end{theorem}
\begin{proof}
If an adversary could forge a capability $c'$ with non-negligible probability, they
could construct a distinguisher for HMAC-SHA256 as a PRF, contradicting the assumed
security of the construction (RFC~2104). The 40-byte input domain is fixed, so no
length-extension attacks apply.
\end{proof}

% -----------------------------------------------------------------------------
\section{Threat Model}
\label{sec:threat}

\VaultOS{} protects against:
\begin{itemize}[leftmargin=2em]
  \item \textbf{Unauthorized data access:} All records encrypted; queries require valid capabilities.
  \item \textbf{Capability forgery:} HMAC-SHA256 seal prevents fabrication.
  \item \textbf{Timing side-channels:} Constant-time HMAC comparison prevents timing attacks.
  \item \textbf{Privilege escalation:} Delegated capabilities are strict subsets of parent rights.
\end{itemize}

Out of scope (MVP): cold-boot attacks, hardware Trojans, DMA attacks, Spectre/Meltdown.

\textbf{Cryptographic RNG dependency.}
The security of all encryption and capability operations depends on the
availability of hardware \texttt{RDRAND} (Intel Ivy Bridge and later).  When
\texttt{RDRAND} is unavailable, VaultOS falls back to a \texttt{RDTSC}-seeded
\texttt{xorshift128+} generator, which provides statistical but \emph{not}
cryptographic randomness.  In virtualised environments where TSC values are
predictable, this fallback weakens all security guarantees---IVs become
predictable and capability tokens lose unforgeability.  The system logs a
warning at boot when operating in fallback mode.

% -----------------------------------------------------------------------------
\section{System Architecture Overview}
\label{sec:arch-overview}

Figure~\ref{fig:architecture} shows the layered architecture.

\begin{figure}[htb]
\centering
\begin{tikzpicture}[
  block/.style={draw,thick,rounded corners=2pt,minimum width=12cm,
    minimum height=1cm,font=\sffamily\small,text=black},
  >=Stealth
]
\node[block,fill=gray!10]                     (hw)  {x86\_64 Hardware: CPU, RAM, PIC, PIT, PS/2 Keyboard \& Mouse, COM1, GOP};
\node[block,fill=blue!8,above=4pt of hw]      (drv) {Drivers: Serial, Framebuffer, Keyboard, Mouse, Font};
\node[block,fill=blue!12,above=4pt of drv]    (mm)  {Memory: PMM (Bitmap) $\mid$ VMM (4-Level Paging) $\mid$ Heap (First-Fit)};
\node[block,fill=blue!18,above=4pt of mm]     (cry) {Crypto: AES-128-CBC $\mid$ SHA-256 $\mid$ HMAC-SHA256 $\mid$ RDRAND RNG};
\node[block,fill=blue!25,above=4pt of cry]    (cap) {Security: Capability Manager (HMAC Tokens, Validation Cache)};
\node[block,fill=blue!32,above=4pt of cap]    (db)  {Database: B-Tree Index $\mid$ Record Store $\mid$ Query Parser $\mid$ Transactions};
\node[block,fill=blue!40,above=4pt of db,
  text=white]                                  (sh)  {VaultShell: Friendly CLI $\mid$ SQL REPL $\mid$ Script Engine $\mid$ Tab Completion};
\node[block,fill=blue!50,above=4pt of sh,
  text=white]                                  (gui) {GUI Desktop: Graphics $\mid$ Events $\mid$ Window Manager $\mid$ Compositor $\mid$ Widgets};

\draw[->,thick] (gui) -- (sh);
\draw[->,thick] (sh) -- (db);
\draw[->,thick] (db) -- (cap);
\draw[->,thick] (cap) -- (cry);
\draw[->,thick] (cry) -- (mm);
\draw[->,thick] (mm) -- (drv);
\draw[->,thick] (drv) -- (hw);
\end{tikzpicture}
\caption{VaultOS layered architecture. Each layer depends only on the layers below it.}
\label{fig:architecture}
\end{figure}

% -----------------------------------------------------------------------------
\section{Comparison with Traditional OS Designs}
\label{sec:comparison}

\begin{table}[htb]
\centering
\caption{VaultOS vs.\ POSIX-based operating systems.}
\label{tab:comparison}
\begin{tabular}{@{}lll@{}}
\toprule
\textbf{Aspect}          & \textbf{POSIX}               & \textbf{VaultOS} \\
\midrule
Resource abstraction     & Files, sockets, pipes        & Database rows \\
Access control           & uid/gid, rwx bits            & HMAC-sealed capabilities \\
Naming                   & Hierarchical paths           & Table + row ID \\
IPC                      & Pipes, signals, shmem        & Message table + queries \\
Process interface        & \texttt{fork/exec/wait}      & \texttt{proc\_create/exit} \\
Encryption               & Optional (dm-crypt, etc.)    & Mandatory, per-table \\
Audit                    & Syslog (optional)            & Built-in AuditTable \\
User interface           & Shell + utilities            & Friendly CLI + SQL REPL + GUI Desktop \\
\bottomrule
\end{tabular}
\end{table}

% =============================================================================
\chapter{System Bootstrap}
\label{ch:boot}

% -----------------------------------------------------------------------------
\section{UEFI Boot Protocol}

The bootloader is a PE32+ EFI application compiled against GNU-EFI. It executes the
following steps:

\begin{enumerate}
\item Initialize the Graphics Output Protocol (GOP) at $1024 \times 768$ resolution.
\item Open the FAT32 system partition and load \texttt{VAULTOS.BIN} at physical address
      \hex{100000} (1\,MiB).
\item Acquire the UEFI memory map (with retry around \texttt{ExitBootServices}).
\item Locate the ACPI RSDP from the EFI configuration tables.
\item Transfer control to \texttt{kernel\_main(BootInfo~*)} at the kernel physical base.
\end{enumerate}

% -----------------------------------------------------------------------------
\section{BootInfo Structure}

The bootloader passes a packed structure containing framebuffer parameters, the UEFI
memory map, kernel location, and runtime service pointers:

\begin{lstlisting}[caption={BootInfo structure (simplified).}]
typedef struct __attribute__((packed)) {
    uint64_t fb_base, fb_width, fb_height, fb_pitch;
    uint32_t fb_bpp, fb_pixel_format;
    uint64_t mmap_base, mmap_size, mmap_desc_size;
    uint32_t mmap_entry_count;
    uint64_t kernel_phys_base, kernel_size;
    uint64_t rsdp_address;
} BootInfo;
\end{lstlisting}

% -----------------------------------------------------------------------------
\section{Kernel Initialization Sequence}
\label{sec:boot-sequence}

The kernel initializes ten subsystems in strict dependency order
(Figure~\ref{fig:boot-flow}).

\begin{figure}[htb]
\centering
\begin{tikzpicture}[
  phase/.style={draw,rounded corners=3pt,minimum width=3.8cm,
    minimum height=0.6cm,font=\sffamily\footnotesize,fill=blue!10},
  arr/.style={->,>=Stealth,thick},
  node distance=0.3cm
]
\node[phase]                    (p1) {1. Serial Init};
\node[phase,right=0.4cm of p1]  (p2) {2. GDT + IDT + PIC};
\node[phase,below=of p1]       (p3) {3. PMM + VMM + Heap};
\node[phase,right=0.4cm of p3]  (p4) {4. FB + PIT + Keyboard};
\node[phase,below=of p3]       (p5) {5. RNG + Crypto Tests};
\node[phase,right=0.4cm of p5]  (p6) {6. Capability System};
\node[phase,below=of p5]       (p7) {7. Database Engine};
\node[phase,right=0.4cm of p7]  (p8) {8. Process + Scheduler};
\node[phase,below=of p7]       (p9) {9. SYSCALL Interface};
\node[phase,right=0.4cm of p9]  (p10){10. VaultShell Launch};

\draw[arr] (p1) -- (p2);
\draw[arr] (p2) |- (p3);
\draw[arr] (p3) -- (p4);
\draw[arr] (p4) |- (p5);
\draw[arr] (p5) -- (p6);
\draw[arr] (p6) |- (p7);
\draw[arr] (p7) -- (p8);
\draw[arr] (p8) |- (p9);
\draw[arr] (p9) -- (p10);
\end{tikzpicture}
\caption{Kernel initialization sequence. Each phase depends on all preceding phases.}
\label{fig:boot-flow}
\end{figure}

% -----------------------------------------------------------------------------
\section{Higher-Half Kernel Design}

The kernel is linked at virtual address \hex{FFFFFFFF80000000} but loaded at physical
address \hex{100000}. This \emph{higher-half} design reserves the lower 128\,TiB of
virtual space for user processes.

\begin{definition}[Virtual Address Space Layout]
\label{def:vas}
The 48-bit canonical virtual address space is partitioned as follows:
\begin{align*}
&\texttt{0x0000000000000000} \text{--} \texttt{0x00007FFFFFFFFFFF} &&\text{User space (128\,TiB)} \\
&\texttt{0xFFFFFFFF80000000} \text{--} \texttt{0xFFFFFFFF81FFFFFF} &&\text{Kernel code/data (2\,MiB)} \\
&\texttt{0xFFFFFFFF82000000} \text{--} \texttt{0xFFFFFFFF91FFFFFF} &&\text{Kernel heap (256\,MiB)} \\
&\texttt{0xFFFFFFFF92000000} \text{--} \texttt{0xFFFFFFFFBFFFFFFF} &&\text{Physical direct map (up to 1\,GiB)} \\
&\texttt{0xFFFFFFFFC0000000} \text{--} \texttt{0xFFFFFFFFCFFFFFFF} &&\text{Framebuffer}
\end{align*}
\end{definition}

% =============================================================================
% PART II: MEMORY MANAGEMENT
% =============================================================================
\part{Memory Management}

% =============================================================================
\chapter{Physical Memory Manager}
\label{ch:pmm}

The physical memory manager (PMM) tracks the availability of 4\,KiB physical pages
using a bitmap data structure.

\section{Bitmap Allocator Design}

\begin{definition}[Page Bitmap]
Let $N = \lfloor M / 4096 \rfloor$ be the number of physical pages, where $M$ is the
total physical memory in bytes. The bitmap $B[0 \ldots \lceil N/64 \rceil - 1]$ is an
array of 64-bit words where bit $i$ of word $B[\lfloor i/64 \rfloor]$ is~1 if page~$i$
is allocated, and~0 if free.
\end{definition}

For 4\,GiB of RAM, $N = 2^{20}$ pages and the bitmap occupies
$2^{20}/8 = 128$\,KiB---a fixed overhead of 0.003\%.

\section{Algorithm: \textsc{Pmm-Alloc}}

\begin{algorithm}[H]
\caption{\textsc{Pmm-Alloc}(): Allocate one physical page.}
\label{alg:pmm-alloc}
\For{$w \leftarrow 0$ \To $\lceil N/64 \rceil - 1$}{
  \If{$B[w] \neq \texttt{0xFFFFFFFFFFFFFFFF}$}{
    $b \leftarrow$ index of lowest clear bit in $B[w]$ \tcp*{\texttt{\_\_builtin\_ctzll(\texttildelow B[w])}}
    $\text{page} \leftarrow 64w + b$ \;
    \If{$\text{page} < N$}{
      set bit $b$ in $B[w]$ \;
      $\text{used\_pages} \leftarrow \text{used\_pages} + 1$ \;
      \Return $\text{page} \times 4096$ \tcp*{physical address}
    }
  }
}
\Return $0$ \tcp*{out of memory}
\end{algorithm}

\section{Algorithm: \textsc{Pmm-Free}}

\begin{algorithm}[H]
\caption{\textsc{Pmm-Free}(\text{phys\_addr}): Free one physical page.}
\label{alg:pmm-free}
$\text{page} \leftarrow \text{phys\_addr} / 4096$ \;
\If{$\text{page} \geq N$ \Or bit $(\text{page} \bmod 64)$ of $B[\lfloor\text{page}/64\rfloor]$ is clear}{
  \Error \text{``double free or invalid page''} \;
}
clear bit $(\text{page} \bmod 64)$ in $B[\lfloor\text{page}/64\rfloor]$ \;
$\text{used\_pages} \leftarrow \text{used\_pages} - 1$ \;
\end{algorithm}

\section{Complexity Analysis}

\begin{theorem}[PMM Allocation Complexity]
\textsc{Pmm-Alloc} runs in $\BigO(N/64)$ worst-case time and $\BigO(1)$ best-case time,
where $N$ is the number of physical pages.
\end{theorem}
\begin{proof}
The outer loop iterates over at most $\lceil N/64 \rceil$ words. Each iteration
performs a constant-time comparison and bit scan. In the best case, the first word
contains a free bit. In the worst case, all words must be scanned.
The per-word bit scan (\texttt{ctzll}) executes in $\BigO(1)$ on x86-64 via the
\texttt{BSF}/\texttt{TZCNT} instruction.
\end{proof}

% =============================================================================
\chapter{Virtual Memory and Paging}
\label{ch:vmm}

\section{x86-64 Four-Level Page Tables}

The x86-64 architecture uses a four-level radix tree to translate 48-bit virtual
addresses to physical addresses.

\begin{definition}[Page Table Entry]
A 64-bit page table entry (PTE) encodes:
\[
\text{PTE} = \underbrace{\text{phys\_addr}[51:12]}_{40\text{ bits}} \;\|\;
             \underbrace{\text{flags}[11:0]}_{12\text{ bits}}
\]
where flags include Present~(P), Writable~(W), User~(U), and No-Execute~(NX, bit~63).
\end{definition}

\begin{definition}[Virtual Address Decomposition]
A 48-bit virtual address $v$ is decomposed as:
\begin{align*}
\text{PML4 index} &= (v \gg 39) \;\&\; \texttt{0x1FF} \\
\text{PDPT index} &= (v \gg 30) \;\&\; \texttt{0x1FF} \\
\text{PD index}   &= (v \gg 21) \;\&\; \texttt{0x1FF} \\
\text{PT index}   &= (v \gg 12) \;\&\; \texttt{0x1FF} \\
\text{Offset}     &= v \;\&\; \texttt{0xFFF}
\end{align*}
\end{definition}

\section{Virtual Address Space Layout}

\begin{figure}[htb]
\centering
\begin{tikzpicture}[
  region/.style={draw,thick,minimum width=9cm,minimum height=0.7cm,
    font=\ttfamily\footnotesize,anchor=west},
]
\node[region,fill=green!10]   at (0,5.6)  {0x0000000000000000 -- User Space (128 TiB)};
\node[region,fill=gray!15]    at (0,4.8)  {... (non-canonical hole) ...};
\node[region,fill=blue!10]    at (0,4.0)  {0xFFFFFFFF80000000 -- Kernel Code/Data (2 MiB)};
\node[region,fill=blue!15]    at (0,3.2)  {0xFFFFFFFF82000000 -- Kernel Heap (256 MiB)};
\node[region,fill=blue!20]    at (0,2.4)  {0xFFFFFFFF92000000 -- Physical Direct Map (1 GiB)};
\node[region,fill=orange!15]  at (0,1.6)  {0xFFFFFFFFC0000000 -- Framebuffer};
\end{tikzpicture}
\caption{Virtual address space layout of the VaultOS kernel.}
\label{fig:vas}
\end{figure}

\section{Algorithm: \textsc{Page-Map}}

\begin{algorithm}[H]
\caption{\textsc{Page-Map}($\text{pml4}$, $\text{virt}$, $\text{phys}$, $\text{flags}$):
Map a virtual page to a physical frame.}
\label{alg:page-map}
$\text{pml4e} \leftarrow \text{pml4}[\text{PML4\_INDEX}(\text{virt})]$ \;
\If{$\text{pml4e}$ is \Not present}{
  allocate a zeroed page for PDPT \;
  $\text{pml4}[\text{PML4\_INDEX}(\text{virt})] \leftarrow \text{pdpt\_phys} \;|\; \text{P} \;|\; \text{W}$ \;
}
$\text{pdpt} \leftarrow$ extract address from pml4e \;
\tcp{Repeat for PDPT $\to$ PD $\to$ PT}
\tcp{(Each level: check present, allocate if needed, descend)}
$\text{pt}[\text{PT\_INDEX}(\text{virt})] \leftarrow \text{phys} \;|\; \text{flags}$ \;
\end{algorithm}

\section{Algorithm: \textsc{Virt-To-Phys}}

\begin{algorithm}[H]
\caption{\textsc{Virt-To-Phys}($\text{pml4}$, $\text{virt}$): Translate virtual to physical address.}
\label{alg:virt-to-phys}
Walk PML4 $\to$ PDPT $\to$ PD $\to$ PT using index functions \;
\If{any level is \Not present}{
  \Return $0$ \tcp*{page not mapped}
}
\If{PD entry has Huge flag (2\,MiB page)}{
  \Return $\text{pd\_phys} \;|\; (\text{virt} \;\&\; \texttt{0x1FFFFF})$ \;
}
\Return $\text{pt\_phys} \;|\; (\text{virt} \;\&\; \texttt{0xFFF})$ \;
\end{algorithm}

\begin{theorem}[Translation Correctness]
For any virtual address $v$ mapped via \textsc{Page-Map} to physical address $p$ with
flags $f$, a subsequent call to \textsc{Virt-To-Phys} returns $p$, provided no
intervening unmap or CR3 reload invalidates the mapping.
\end{theorem}
\begin{proof}[Proof sketch]
\textsc{Page-Map}$(v, p, f)$ writes the entry $(\text{base}=p,\;\text{flags}=f)$ into
the page-table entry (PTE) selected by the level-1 index of~$v$.  The four-level
walk in \textsc{Virt-To-Phys} extracts exactly the same indices from~$v$
(bits~47--39, 38--30, 29--21, 20--12) and follows the same PML4$\to$PDP$\to$PD$\to$PT
chain.  At the final level it reads the PTE base address, which is~$p$.
Since x86-64 address translation is deterministic for a given CR3 and page-table
contents, the result follows.  The ``no intervening unmap'' proviso ensures the
PTE has not been cleared between the two calls.
\end{proof}

% =============================================================================
\chapter{Kernel Heap Allocator}
\label{ch:heap}

\section{First-Fit with Coalescing}

The kernel heap is a doubly-linked list of blocks, each preceded by a header:

\begin{lstlisting}[caption={Heap block header.}]
typedef struct heap_block {
    uint64_t     magic;    /* 0xDEADBEEF */
    size_t       size;     /* usable data bytes */
    bool         free;
    heap_block_t *next, *prev;
} heap_block_t;
\end{lstlisting}

\section{Algorithm: \textsc{Kmalloc}}

\begin{algorithm}[H]
\caption{\textsc{Kmalloc}($n$): Allocate $n$ bytes from the kernel heap.}
\label{alg:kmalloc}
$n \leftarrow \textsc{Align-Up}(n, 16)$ \;
$b \leftarrow \text{head of block list}$ \;
\While{$b \neq \Nil$}{
  \If{$b.\field{free}$ \And $b.\field{size} \geq n$}{
    \If{$b.\field{size} - n > \text{sizeof(header)} + 32$}{
      \textsc{Split}($b$, $n$) \tcp*{create free block from remainder}
    }
    $b.\field{free} \leftarrow \False$ \;
    \Return pointer to $b$'s data area \;
  }
  $b \leftarrow b.\field{next}$ \;
}
\Return \Nil \tcp*{out of memory}
\end{algorithm}

\section{Algorithm: \textsc{Kfree} with Coalescing}

\begin{algorithm}[H]
\caption{\textsc{Kfree}($p$): Free a previously allocated block and coalesce neighbors.}
\label{alg:kfree}
$b \leftarrow$ block header preceding $p$ \;
\textbf{assert} $b.\field{magic} = \texttt{0xDEADBEEF}$ \;
$b.\field{free} \leftarrow \True$ \;
\tcp{Forward coalescing}
\If{$b.\field{next} \neq \Nil$ \And $b.\field{next}.\field{free}$}{
  $b.\field{size} \leftarrow b.\field{size} + \text{sizeof(header)} + b.\field{next}.\field{size}$ \;
  $b.\field{next} \leftarrow b.\field{next}.\field{next}$ \;
  \If{$b.\field{next} \neq \Nil$}{$b.\field{next}.\field{prev} \leftarrow b$}
}
\tcp{Backward coalescing}
\If{$b.\field{prev} \neq \Nil$ \And $b.\field{prev}.\field{free}$}{
  $b.\field{prev}.\field{size} \leftarrow b.\field{prev}.\field{size} + \text{sizeof(header)} + b.\field{size}$ \;
  $b.\field{prev}.\field{next} \leftarrow b.\field{next}$ \;
  \If{$b.\field{next} \neq \Nil$}{$b.\field{next}.\field{prev} \leftarrow b.\field{prev}$}
}
\end{algorithm}

\begin{theorem}[Heap Invariant]
After any sequence of \textsc{Kmalloc} and \textsc{Kfree} operations, no two adjacent
blocks in the free list are both free.
\end{theorem}
\begin{proof}
\textsc{Kfree} explicitly coalesces with both the predecessor and successor blocks.
If either neighbor is free, the blocks are merged. Therefore, upon return from
\textsc{Kfree}, the freed block has no free neighbor, maintaining the invariant.
\textsc{Kmalloc} can only split a free block into (allocated, free), which cannot
create adjacent free blocks.
\end{proof}

% =============================================================================
% PART III: CRYPTOGRAPHIC PRIMITIVES
% =============================================================================
\part{Cryptographic Primitives}

% =============================================================================
\chapter{SHA-256}
\label{ch:sha256}

SHA-256 is the foundation of \VaultOS's integrity guarantees. It is used in HMAC for
capability sealing and in key derivation for per-table encryption.

\section{Merkle-Damg\r{a}rd Construction}

SHA-256 follows the Merkle-Damg\r{a}rd paradigm: the message is padded to a multiple
of 512 bits, then processed in 512-bit (64-byte) blocks. Each block is compressed into
the running 256-bit state.

\begin{definition}[SHA-256 State]
The state consists of eight 32-bit words $H_0, \ldots, H_7$ initialized to the
fractional parts of the square roots of the first eight primes:
\[
H_0 = \texttt{6a09e667},\quad H_1 = \texttt{bb67ae85},\quad \ldots,\quad
H_7 = \texttt{5be0cd19}
\]
\end{definition}

\section{Compression Function}

\begin{algorithm}[H]
\caption{\textsc{SHA256-Transform}($H[0..7]$, $\text{block}[0..63]$):
Process one 512-bit block.}
\label{alg:sha256}
\tcp{Message schedule}
\For{$i \leftarrow 0$ \To $15$}{
  $W[i] \leftarrow$ big-endian 32-bit word from $\text{block}[4i \ldots 4i{+}3]$ \;
}
\For{$i \leftarrow 16$ \To $63$}{
  $W[i] \leftarrow \sigma_1(W[i{-}2]) + W[i{-}7] + \sigma_0(W[i{-}15]) + W[i{-}16]$ \;
}
\tcp{Initialize working variables}
$a,b,c,d,e,f,g,h \leftarrow H[0], H[1], \ldots, H[7]$ \;
\tcp{Compression rounds}
\For{$i \leftarrow 0$ \To $63$}{
  $T_1 \leftarrow h + \Sigma_1(e) + \mathrm{Ch}(e,f,g) + K_i + W[i]$ \;
  $T_2 \leftarrow \Sigma_0(a) + \mathrm{Maj}(a,b,c)$ \;
  $h \leftarrow g;\; g \leftarrow f;\; f \leftarrow e;\; e \leftarrow d + T_1$ \;
  $d \leftarrow c;\; c \leftarrow b;\; b \leftarrow a;\; a \leftarrow T_1 + T_2$ \;
}
\tcp{Update state}
$H[i] \leftarrow H[i] + \{a,b,c,d,e,f,g,h\}_i$ for $i = 0,\ldots,7$ \;
\end{algorithm}

\noindent
The mixing functions are defined as:
\begin{align*}
\mathrm{Ch}(x,y,z) &= (x \wedge y) \oplus (\neg x \wedge z) \\
\mathrm{Maj}(x,y,z) &= (x \wedge y) \oplus (x \wedge z) \oplus (y \wedge z) \\
\Sigma_0(x) &= \mathrm{ROTR}^2(x) \oplus \mathrm{ROTR}^{13}(x) \oplus \mathrm{ROTR}^{22}(x) \\
\Sigma_1(x) &= \mathrm{ROTR}^6(x) \oplus \mathrm{ROTR}^{11}(x) \oplus \mathrm{ROTR}^{25}(x) \\
\sigma_0(x) &= \mathrm{ROTR}^7(x) \oplus \mathrm{ROTR}^{18}(x) \oplus (x \gg 3) \\
\sigma_1(x) &= \mathrm{ROTR}^{17}(x) \oplus \mathrm{ROTR}^{19}(x) \oplus (x \gg 10)
\end{align*}

\section{Block-Based Update}

\begin{algorithm}[H]
\caption{\textsc{SHA256-Update}(\text{ctx}, \text{data}, \text{len}):
Incrementally feed data to the hash function.}
\label{alg:sha256-update}
$\text{buffered} \leftarrow \text{ctx.count} \bmod 64$ \;
$\text{ctx.count} \leftarrow \text{ctx.count} + \text{len}$ \;
\If{$\text{buffered} > 0$}{
  $\text{need} \leftarrow 64 - \text{buffered}$ \;
  \If{$\text{len} < \text{need}$}{
    copy \text{data} to buffer at offset \text{buffered} \;
    \Return \;
  }
  copy \text{need} bytes to complete buffer \;
  \textsc{SHA256-Transform}(\text{ctx.state}, \text{ctx.buffer}) \;
  advance data by \text{need};\; $\text{len} \leftarrow \text{len} - \text{need}$ \;
}
\While{$\text{len} \geq 64$}{
  \textsc{SHA256-Transform}(\text{ctx.state}, \text{data}) \tcp*{process directly}
  advance data by 64;\; $\text{len} \leftarrow \text{len} - 64$ \;
}
\If{$\text{len} > 0$}{
  copy remaining \text{len} bytes to ctx.buffer \;
}
\end{algorithm}

\begin{observation}
The block-based update processes aligned 64-byte blocks directly from the input buffer,
avoiding $64\times$ per-byte overhead compared to a na\"ive byte-at-a-time approach.
For HMAC computation of a 64-byte ipad, this processes the entire block in a single
call to \textsc{SHA256-Transform}.
\end{observation}

% =============================================================================
\chapter{AES-128}
\label{ch:aes}

AES-128 provides confidentiality for all database records. \VaultOS{} implements both
a software path with precomputed lookup tables and a hardware-accelerated path using
AES-NI instructions.

\section{Rijndael Cipher Structure}

AES-128 operates on 128-bit (16-byte) blocks using a 128-bit key, performing 10 rounds
of transformations on a $4 \times 4$ byte matrix called the \emph{state}.

\begin{definition}[AES Round]
Each round (except the last) applies four transformations:
\[
\text{Round}(s, k_r) = \textsc{AddRoundKey}\bigl(
  \textsc{MixColumns}\bigl(
    \textsc{ShiftRows}\bigl(
      \textsc{SubBytes}(s)
    \bigr)
  \bigr), k_r\bigr)
\]
The final round omits \textsc{MixColumns}.
\end{definition}

\section{$\mathrm{GF}(2^8)$ Arithmetic and Precomputed Tables}

The \textsc{MixColumns} step requires multiplication in $\mathrm{GF}(2^8)$ with the
irreducible polynomial $x^8 + x^4 + x^3 + x + 1$ (= \hex{11B}).

Rather than computing these multiplications at runtime (8 iterations per byte),
\VaultOS{} uses six precomputed 256-byte lookup tables:

\begin{center}
\begin{tabular}{@{}cl@{}}
\toprule
\textbf{Table} & \textbf{Usage} \\
\midrule
$\text{mul2}[256]$ & \textsc{MixColumns}: $\{2\} \cdot x$ \\
$\text{mul3}[256]$ & \textsc{MixColumns}: $\{3\} \cdot x$ \\
$\text{mul9}[256]$ & \textsc{InvMixColumns}: $\{9\} \cdot x$ \\
$\text{mul11}[256]$ & \textsc{InvMixColumns}: $\{b\} \cdot x$ \\
$\text{mul13}[256]$ & \textsc{InvMixColumns}: $\{d\} \cdot x$ \\
$\text{mul14}[256]$ & \textsc{InvMixColumns}: $\{e\} \cdot x$ \\
\bottomrule
\end{tabular}
\end{center}

\noindent
This replaces $48 \times 8 = 384$ loop iterations per block with 48 table lookups.

\section{Algorithm: \textsc{AES-Key-Expand}}

\begin{algorithm}[H]
\caption{\textsc{AES-Key-Expand}($\text{key}[0..15]$): Expand 128-bit key to 11 round keys.}
\label{alg:aes-key}
$w[0..3] \leftarrow$ 32-bit words from key \;
\For{$i \leftarrow 4$ \To $43$}{
  $\text{temp} \leftarrow w[i-1]$ \;
  \If{$i \bmod 4 = 0$}{
    $\text{temp} \leftarrow \textsc{SubWord}(\textsc{RotWord}(\text{temp})) \oplus \text{Rcon}[i/4]$ \;
  }
  $w[i] \leftarrow w[i-4] \oplus \text{temp}$ \;
}
\end{algorithm}

\section{Algorithm: \textsc{AES-Encrypt-Block}}

\begin{algorithm}[H]
\caption{\textsc{AES-Encrypt-Block}($\text{ctx}$, $\text{block}[0..15]$): Encrypt one 128-bit block.}
\label{alg:aes-enc}
\textsc{AddRoundKey}(block, $\text{ctx.rk}[0]$) \;
\For{$r \leftarrow 1$ \To $9$}{
  \textsc{SubBytes}(block) \tcp*{S-box lookup, 16 bytes}
  \textsc{ShiftRows}(block) \tcp*{cyclic row rotations}
  \textsc{MixColumns}(block) \tcp*{GF($2^8$) via lookup tables}
  \textsc{AddRoundKey}(block, $\text{ctx.rk}[r]$) \;
}
\textsc{SubBytes}(block) \;
\textsc{ShiftRows}(block) \;
\textsc{AddRoundKey}(block, $\text{ctx.rk}[10]$) \;
\end{algorithm}

\section{CBC Mode}

\begin{algorithm}[H]
\caption{\textsc{AES-CBC-Encrypt}($\text{ctx}$, $\text{iv}$, $\text{pt}$, $\text{ct}$, $\text{len}$)}
\label{alg:cbc-enc}
$\text{prev} \leftarrow \text{iv}$ \;
\For{$\text{off} \leftarrow 0$ \To $\text{len} - 16$ \textbf{step} $16$}{
  \For{$i \leftarrow 0$ \To $15$}{
    $\text{ct}[\text{off}+i] \leftarrow \text{pt}[\text{off}+i] \oplus \text{prev}[i]$ \;
  }
  \textsc{AES-Encrypt-Block}(\text{ctx}, $\text{ct} + \text{off}$) \;
  $\text{prev} \leftarrow \text{ct} + \text{off}$ \tcp*{pointer, no copy}
}
\end{algorithm}

\begin{observation}
The CBC encrypt implementation avoids redundant \texttt{memcpy} by maintaining a
pointer to the previous ciphertext block rather than copying it to a temporary buffer.
This eliminates 2 of 3 \texttt{memcpy} calls per block.
\end{observation}

\section{AES-NI Hardware Acceleration}

When the CPU supports AES-NI (detected via \texttt{CPUID.01H:ECX[25]}), \VaultOS{}
dispatches to a hardware-accelerated path:

\begin{lstlisting}[caption={AES-NI encrypt (inline assembly sketch).},language={[x86masm]Assembler}]
movdqu  (%block), %xmm0      ; Load plaintext block
pxor    0(%rk),   %xmm0      ; AddRoundKey 0
aesenc  16(%rk),  %xmm0      ; Rounds 1-9 (9 instructions)
aesenc  32(%rk),  %xmm0
; ... (rounds 3-9) ...
aesenclast 160(%rk), %xmm0   ; Final round
movdqu  %xmm0,   (%block)    ; Store ciphertext
\end{lstlisting}

For decryption, each intermediate round key must be transformed via \texttt{AESIMC}
(InvMixColumns) before use with \texttt{AESDEC}.

\begin{table}[htb]
\centering
\caption{AES-128-CBC performance: 1\,KiB $\times$ 100 iterations.}
\label{tab:aes-perf}
\begin{tabular}{@{}lrrr@{}}
\toprule
\textbf{Operation} & \textbf{AES-NI (Haswell)} & \textbf{Software (Nehalem)} & \textbf{Speedup} \\
\midrule
Encrypt & 29{,}388 cycles/op  & 217{,}929 cycles/op  & $7.4\times$ \\
Decrypt & 39{,}023 cycles/op  & 284{,}357 cycles/op  & $7.3\times$ \\
\bottomrule
\end{tabular}
\end{table}

\begin{theorem}[CBC IND-CPA Security]
\label{thm:cbc}
AES-128-CBC with random IVs is IND-CPA secure (indistinguishable under chosen-plaintext
attack) assuming AES is a pseudorandom permutation (PRP), up to $2^{64}$ blocks
(the birthday bound on 128-bit blocks).
\end{theorem}

% =============================================================================
\chapter{HMAC-SHA256 and Random Number Generation}
\label{ch:hmac}

\section{HMAC Construction}

HMAC-SHA256 (RFC~2104) provides message authentication:
\[
\text{HMAC}(K, m) = \text{SHA256}\bigl((K \oplus \text{opad}) \;\|\;
  \text{SHA256}((K \oplus \text{ipad}) \;\|\; m)\bigr)
\]
where $\text{ipad} = \texttt{0x36}^{64}$ and $\text{opad} = \texttt{0x5C}^{64}$.

\section{Pre-computed Context}

Since the master key $K$ is fixed at boot, \VaultOS{} pre-computes the SHA-256 state
after processing the ipad and opad blocks:

\begin{algorithm}[H]
\caption{\textsc{HMAC-Init}($\text{ctx}$, $K$, $\text{klen}$): Pre-compute HMAC state for key $K$.}
\label{alg:hmac-init}
\If{$\text{klen} > 64$}{
  $K' \leftarrow \textsc{SHA256}(K)$;\quad pad to 64 bytes with zeros \;
}
\Else{
  $K' \leftarrow K$ padded to 64 bytes with zeros \;
}
$\text{ipad}[i] \leftarrow K'[i] \oplus \texttt{0x36}$ for $i = 0, \ldots, 63$ \;
$\text{opad}[i] \leftarrow K'[i] \oplus \texttt{0x5C}$ for $i = 0, \ldots, 63$ \;
$\text{ctx.inner\_base} \leftarrow \textsc{SHA256-Update}(\textsc{SHA256-Init}(), \text{ipad}, 64)$ \;
$\text{ctx.outer\_base} \leftarrow \textsc{SHA256-Update}(\textsc{SHA256-Init}(), \text{opad}, 64)$ \;
\end{algorithm}

\begin{algorithm}[H]
\caption{\textsc{HMAC-Compute}($\text{ctx}$, $\text{data}$, $\text{len}$, $\text{mac}$):
Compute HMAC using pre-computed context.}
\label{alg:hmac-compute}
$\text{inner} \leftarrow \text{clone}(\text{ctx.inner\_base})$ \;
\textsc{SHA256-Update}(inner, data, len) \;
\textsc{SHA256-Final}(inner, inner\_hash) \;
$\text{outer} \leftarrow \text{clone}(\text{ctx.outer\_base})$ \;
\textsc{SHA256-Update}(outer, inner\_hash, 32) \;
\textsc{SHA256-Final}(outer, mac) \;
\end{algorithm}

\begin{observation}
Pre-computing the HMAC context eliminates re-hashing the 128 bytes of ipad and opad
for every HMAC computation. For capability validation (40-byte payload), this reduces
SHA-256 block processing from 4 blocks to 2 blocks per HMAC---a $\sim\!3\times$ speedup.
\end{observation}

\section{Constant-Time Verification}

\begin{algorithm}[H]
\caption{\textsc{HMAC-Verify}($a[0..n{-}1]$, $b[0..n{-}1]$): Constant-time comparison.}
\label{alg:hmac-verify}
$\text{diff} \leftarrow 0$ \;
\For{$i \leftarrow 0$ \To $n - 1$}{
  $\text{diff} \leftarrow \text{diff} \;|\; (a[i] \oplus b[i])$ \;
}
\Return $\text{diff} = 0$ \;
\end{algorithm}

The OR-accumulation ensures the loop runs in exactly $n$ iterations regardless of where
mismatches occur, preventing timing side-channel attacks.

\section{Random Number Generation}

\VaultOS{} seeds its entropy pool from the \texttt{RDRAND} instruction (when available)
and falls back to \texttt{RDTSC}-based seeding with an xorshift128+ PRNG.

\begin{theorem}[HMAC Unforgeability]
Under the assumption that SHA-256 is a pseudorandom function (PRF) when keyed,
HMAC-SHA256 is $(t, q, \epsilon)$-unforgeable: no adversary running in time $t$ and
making $q$ queries can forge a valid MAC with probability greater than
$\epsilon + q \cdot 2^{-256}$.
\end{theorem}

% =============================================================================
% PART IV: DATA STRUCTURES
% =============================================================================
\part{Data Structures}

% =============================================================================
\chapter{B-Tree Index}
\label{ch:btree}

The database engine indexes every table with a B-tree of order $b = 64$,
providing $\BigO(\log_{b} n)$ search, insert, and delete operations.

\section{B-Tree Properties}

\begin{definition}[B-Tree of Order $b$]
\label{def:btree-order}
A B-tree of order $b$ (maximum branching factor) satisfies:
\begin{enumerate}
\item Every node has at most $b$ children and $b - 1$ keys ($= 63$ for $b = 64$).
\item Every non-root internal node has at least $\lceil b/2 \rceil$ children
      ($= 32$) and $\lceil b/2 \rceil - 1$ keys ($= 31$).
\item The root has at least $2$ children (if non-leaf) or $0$ keys (if empty tree).
\item A node with $k$ keys has $k + 1$ children (if internal).
\item All leaves appear at the same depth.
\end{enumerate}
\end{definition}

\noindent
In \VaultOS{}, \texttt{BTREE\_ORDER} $= 64$, so nodes store up to 63 keys and 64 child
pointers.  Each node occupies approximately $63 \times 8 + 63 \times 8 + 64 \times 8 + 8
= 1{,}520$ bytes, fitting comfortably in L1 cache.

\begin{remark}
Our order~$b$ corresponds to CLRS \emph{minimum degree} $t = b/2 = 32$: CLRS
requires $t - 1 \leq k \leq 2t - 1$ keys per non-root node, giving $31 \leq k \leq 63$,
which matches our $\lceil b/2 \rceil - 1 \leq k \leq b - 1$.
Some references (Knuth) define order as the maximum number of keys; ours defines it
as the maximum number of children.
\end{remark}

\section{Node Structure}

\begin{lstlisting}[caption={B-tree node structure.}]
typedef struct btree_node {
    uint64_t   keys[63];       /* sorted key array */
    void      *values[63];     /* associated record pointers */
    btree_node *children[64];  /* child pointers */
    uint32_t   num_keys;
    bool       is_leaf;
} btree_node_t;
\end{lstlisting}

\section{Algorithm: \textsc{B-Tree-Search}}

\begin{algorithm}[H]
\caption{\textsc{B-Tree-Search}($x$, $k$): Search for key $k$ in subtree rooted at $x$.}
\label{alg:btree-search}
$i \leftarrow 0$ \;
\While{$i < x.\field{num\_keys}$ \And $k > x.\field{keys}[i]$}{
  $i \leftarrow i + 1$ \;
}
\If{$i < x.\field{num\_keys}$ \And $k = x.\field{keys}[i]$}{
  \Return $x.\field{values}[i]$ \;
}
\If{$x.\field{is\_leaf}$}{
  \Return \Nil \;
}
\Return \textsc{B-Tree-Search}($x.\field{children}[i]$, $k$) \;
\end{algorithm}

\section{Algorithm: \textsc{B-Tree-Insert}}

\begin{algorithm}[H]
\caption{\textsc{B-Tree-Insert}($T$, $k$, $v$): Insert key $k$ with value $v$ into B-tree $T$.}
\label{alg:btree-insert}
$r \leftarrow T.\field{root}$ \;
\If{$r.\field{num\_keys} = 63$}{
  $s \leftarrow$ new node (leaf $=$ \False) \;
  $s.\field{children}[0] \leftarrow r$ \;
  $T.\field{root} \leftarrow s$ \;
  \textsc{B-Tree-Split-Child}($s$, $0$) \;
  \textsc{B-Tree-Insert-Nonfull}($s$, $k$, $v$) \;
}
\Else{
  \textsc{B-Tree-Insert-Nonfull}($r$, $k$, $v$) \;
}
\end{algorithm}

\section{Algorithm: \textsc{B-Tree-Split-Child}}

\begin{algorithm}[H]
\caption{\textsc{B-Tree-Split-Child}($x$, $i$): Split full child $x.\field{children}[i]$.}
\label{alg:btree-split}
$y \leftarrow x.\field{children}[i]$ \tcp*{full child ($y.\field{num\_keys} = 63$)}
$z \leftarrow$ new node ($z.\field{is\_leaf} \leftarrow y.\field{is\_leaf}$) \;
$\text{mid} \leftarrow 31$ \tcp*{median index}
\tcp{Copy upper half of $y$ to $z$}
\For{$j \leftarrow 0$ \To $30$}{
  $z.\field{keys}[j] \leftarrow y.\field{keys}[\text{mid}+1+j]$ \;
  $z.\field{values}[j] \leftarrow y.\field{values}[\text{mid}+1+j]$ \;
}
\If{\Not $y.\field{is\_leaf}$}{
  \For{$j \leftarrow 0$ \To $31$}{
    $z.\field{children}[j] \leftarrow y.\field{children}[\text{mid}+1+j]$ \;
  }
}
$z.\field{num\_keys} \leftarrow 31$ \;
$y.\field{num\_keys} \leftarrow 31$ \;
\tcp{Promote median key to parent}
shift $x.\field{keys}[i..], x.\field{children}[i{+}1..]$ right by 1 \;
$x.\field{keys}[i] \leftarrow y.\field{keys}[\text{mid}]$ \;
$x.\field{values}[i] \leftarrow y.\field{values}[\text{mid}]$ \;
$x.\field{children}[i+1] \leftarrow z$ \;
$x.\field{num\_keys} \leftarrow x.\field{num\_keys} + 1$ \;
\end{algorithm}

\begin{figure}[htb]
\centering
\begin{tikzpicture}[
  bnode/.style={draw,thick,rounded corners=2pt,minimum height=0.6cm,
    inner sep=3pt,font=\footnotesize},
  level distance=1.5cm,
  sibling distance=4cm,
  edge from parent/.style={draw,thick,->,>=Stealth},
]
\node[bnode] (before) {$k_1 \;|\; k_2 \;|\; \cdots \;|\; k_{63}$}
  [grow=down]
  child { node[bnode] {child 0} }
  child { node[bnode] {child 1} };
\node[right=4cm of before,font=\Large] {$\Longrightarrow$};
\node[bnode,right=6cm of before] (parent) {$k_{32}$}
  [grow=down]
  child { node[bnode] {$k_1 \;|\; \cdots \;|\; k_{31}$} }
  child { node[bnode] {$k_{33} \;|\; \cdots \;|\; k_{63}$} };
\node[above=0.3cm of before,font=\small\itshape] {Before split};
\node[above=0.3cm of parent,font=\small\itshape] {After split};
\end{tikzpicture}
\caption{B-tree node split: the median key is promoted to the parent.}
\label{fig:btree-split}
\end{figure}

\begin{theorem}[B-Tree Height Bound]
\label{thm:btree-height}
A B-tree of order $b = 64$ (CLRS minimum degree $t = b/2 = 32$) containing $n$ keys has
height $h \leq \log_{t}\!\bigl(\frac{n+1}{2}\bigr) = \log_{32}\!\bigl(\frac{n+1}{2}\bigr)$.
\end{theorem}
\begin{proof}
The root has at least $1$ key and $2$ children.  Every other internal node has at
least $t = 32$ children and $t - 1 = 31$ keys.  At depth~$d \geq 1$ there are at
least $2t^{d-1}$ nodes, each with at least $t - 1$ keys.  Summing:
$n \geq 1 + 2(t-1)\sum_{i=0}^{h-1} t^i = 2t^h - 1$, yielding
$h \leq \log_t\!\bigl(\frac{n+1}{2}\bigr)$.
\end{proof}

\begin{corollary}
For $n = 10^6$ keys, $h \leq 1 + \log_{32}(500{,}000) \approx 4.8$, so the tree has at
most 5 levels. For $n = 10^3$ (typical VaultOS workload), $h \leq 3$.
\end{corollary}

\section{Algorithm: \textsc{B-Tree-Delete} (Lazy)}

\begin{algorithm}[H]
\caption{\textsc{B-Tree-Delete}($T$, $k$): Delete key $k$ from B-tree $T$ (lazy MVP).}
\label{alg:btree-delete}
$x \leftarrow T.\field{root}$ \;
\While{$x \neq \textsc{NULL}$}{
  $i \leftarrow 0$ \;
  \While{$i < x.\field{num\_keys}$ \And $k > x.\field{keys}[i]$}{
    $i \leftarrow i + 1$ \;
  }
  \If{$i < x.\field{num\_keys}$ \And $k = x.\field{keys}[i]$}{
    \tcp{Key found at position $i$}
    \If{$x.\field{is\_leaf}$}{
      \tcp{Case 1: Leaf node --- shift remaining keys left}
      \For{$j \leftarrow i$ \KwTo $x.\field{num\_keys} - 2$}{
        $x.\field{keys}[j] \leftarrow x.\field{keys}[j+1]$ \;
        $x.\field{values}[j] \leftarrow x.\field{values}[j+1]$ \;
      }
      $x.\field{num\_keys} \leftarrow x.\field{num\_keys} - 1$ \;
      \Return \textsc{TRUE} \;
    }
    \Else{
      \tcp{Case 2: Internal node --- mark value as deleted}
      $x.\field{values}[i] \leftarrow \textsc{NULL}$ \;
      \Return \textsc{TRUE} \;
    }
  }
  \If{$x.\field{is\_leaf}$}{
    \Return \textsc{FALSE} \tcp*{key not in tree}
  }
  $x \leftarrow x.\field{children}[i]$ \tcp*{descend to child}
}
\Return \textsc{FALSE} \;
\end{algorithm}

\begin{remark}[Lazy Delete Limitations]
Algorithm~\ref{alg:btree-delete} is a simplified MVP implementation that does
\emph{not} maintain the B-tree minimum-occupancy invariant
(Definition~\ref{def:btree-order}, property~2).  Specifically:
\begin{itemize}[leftmargin=2em]
  \item \textbf{Leaf nodes} may underflow below $\lceil b/2 \rceil - 1 = 31$ keys
        after deletion.
  \item \textbf{Internal nodes} retain tombstoned entries (\textsc{NULL} values)
        rather than replacing the key with an in-order predecessor/successor.
  \item No merge or key redistribution between siblings is performed.
\end{itemize}
For the current VaultOS workload (six system tables with low churn), this is
acceptable.  A full CLRS-compliant delete with merge and redistribute is planned
for a future session.
\end{remark}

% =============================================================================
\chapter{Auxiliary Data Structures}
\label{ch:aux}

\section{Intrusive Doubly-Linked Lists}

\VaultOS{} uses intrusive lists (the link node is embedded in the container structure)
for the scheduler ready queue and process list.

\begin{lstlisting}[caption={Intrusive list node and macros.}]
typedef struct list_node {
    struct list_node *next, *prev;
} list_node_t;

#define container_of(ptr, type, member) \
    ((type *)((char *)(ptr) - offsetof(type, member)))
\end{lstlisting}

All list operations (insert head/tail, remove, iterate) run in $\BigO(1)$ time.

\section{Bitmap Operations}

The bitmap module provides bit-level operations used by the PMM:

\begin{itemize}
\item \textsc{Bitmap-Set}($B$, $i$): Set bit $i$ in $\BigO(1)$.
\item \textsc{Bitmap-Clear}($B$, $i$): Clear bit $i$ in $\BigO(1)$.
\item \textsc{Bitmap-Test}($B$, $i$): Test bit $i$ in $\BigO(1)$.
\item \textsc{Bitmap-Find-Clear}($B$, $n$, $\text{start}$): Find first clear bit $\geq \text{start}$ in $\BigO(n/64)$.
\end{itemize}

\section{Ring Buffers}

The keyboard driver and IPC subsystem use fixed-size circular buffers:

\begin{definition}[Ring Buffer]
A ring buffer of capacity $C$ uses indices $\text{head}$ and $\text{tail}$ in $[0, C)$.
The buffer is empty when $\text{head} = \text{tail}$ and full when
$(\text{head} + 1) \bmod C = \text{tail}$.
Enqueue and dequeue are $\BigO(1)$.
\end{definition}

% =============================================================================
% PART V: SECURITY ARCHITECTURE
% =============================================================================
\part{Security Architecture}

% =============================================================================
\chapter{Capability System}
\label{ch:capabilities}

\section{Capability Token Structure}

Each capability is a 96-byte structure (Definition~\ref{sec:cap-model}) stored in a
direct-indexed array of 1{,}024 slots. The capability ID serves as the array index
(offset by 1), enabling $\BigO(1)$ lookup.

\section{Algorithm: \textsc{Cap-Create}}

\begin{algorithm}[H]
\caption{\textsc{Cap-Create}($\text{obj\_id}$, $\text{type}$, $\text{owner}$, $\text{rights}$, $\text{parent}$):
Create and seal a new capability.}
\label{alg:cap-create}
$c.\field{cap\_id} \leftarrow \text{next\_cap\_id}$;\quad $\text{next\_cap\_id} \leftarrow \text{next\_cap\_id} + 1$ \;
$c.\field{obj\_id} \leftarrow \text{obj\_id}$;\quad $c.\field{type} \leftarrow \text{type}$ \;
$c.\field{owner} \leftarrow \text{owner}$;\quad $c.\field{rights} \leftarrow \text{rights}$ \;
$c.\field{parent} \leftarrow \text{parent}$;\quad $c.\field{revoked} \leftarrow \False$ \;
\tcp{Seal with HMAC}
$\text{data} \leftarrow c.\field{cap\_id} \| c.\field{obj\_id} \| c.\field{owner} \| c.\field{rights} \| c.\field{type} \| c.\field{parent}$ \;
$c.\field{hmac} \leftarrow \textsc{HMAC-Compute}(\text{master\_ctx}, \text{data}, 40)$ \;
\Return $c$ \;
\end{algorithm}

\section{Algorithm: \textsc{Cap-Validate} with Cache}

\begin{algorithm}[H]
\caption{\textsc{Cap-Validate}($c$): Verify capability integrity using cache.}
\label{alg:cap-validate}
\If{$c.\field{revoked}$}{\Return \False}
\If{$c.\field{expires\_at} \neq 0$ \And $\text{now} > c.\field{expires\_at}$}{\Return \False}
\tcp{Check validation cache}
$\text{idx} \leftarrow c.\field{cap\_id} \bmod 64$ \;
\If{$\text{cache}[\text{idx}].\field{occupied}$ \And $\text{cache}[\text{idx}].\field{cap\_id} = c.\field{cap\_id}$
    \And $\text{now} - \text{cache}[\text{idx}].\field{validated\_at} < 1000$}{
  \Return $\text{cache}[\text{idx}].\field{valid}$ \tcp*{cache hit}
}
\tcp{Cache miss: recompute HMAC}
$c' \leftarrow c$ \;
\textsc{Cap-Compute-HMAC}($c'$) \;
$\text{valid} \leftarrow \textsc{HMAC-Verify}(c.\field{hmac}, c'.\field{hmac}, 32)$ \;
$\text{cache}[\text{idx}] \leftarrow (c.\field{cap\_id}, \text{now}, \text{valid}, \True)$ \;
\Return valid \;
\end{algorithm}

\begin{observation}
The validation cache uses a direct-mapped scheme with 64 entries and a 1-second TTL.
Under typical workloads (repeated access to the same capabilities), this eliminates
$\sim\!95\%$ of HMAC recomputations.
\end{observation}

\section{Algorithm: \textsc{Cap-Check}}

\begin{algorithm}[H]
\caption{\textsc{Cap-Check}($\text{pid}$, $\text{obj\_id}$, $\text{required\_rights}$):
Check if process has required rights on object.}
\label{alg:cap-check}
\If{$\text{pid} = 0$}{\Return \True \tcp*{kernel always authorized}}
\For{$i \leftarrow 1$ \To $\text{next\_cap\_id} - 1$}{
  $c \leftarrow \textsc{Cap-Table-Lookup}(i)$ \tcp*{$\BigO(1)$ direct index}
  \If{$c = \Nil$ \Or $c.\field{owner} \neq \text{pid}$}{continue}
  \If{$c.\field{obj\_id} \neq \text{obj\_id}$ \And $c.\field{type} \neq \const{System}$}{continue}
  \If{$(c.\field{rights} \;\&\; \text{required\_rights}) \neq \text{required\_rights}$}{continue}
  \If{\textsc{Cap-Validate}($c$)}{\Return \True}
}
\Return \False \;
\end{algorithm}

\section{Delegation and Revocation}

\begin{algorithm}[H]
\caption{\textsc{Cap-Revoke}($\text{owner\_pid}$, $\text{cap\_id}$): Revoke a capability
and cascade to children.}
\label{alg:cap-revoke}
$c \leftarrow \textsc{Cap-Table-Lookup}(\text{cap\_id})$ \;
\If{$c = \Nil$}{\Return \const{NotFound}}
$c.\field{revoked} \leftarrow \True$ \;
invalidate cache entry for cap\_id \;
\tcp{Cascade: revoke all children}
\For{$i \leftarrow 1$ \To $\text{next\_cap\_id} - 1$}{
  $\text{child} \leftarrow \textsc{Cap-Table-Lookup}(i)$ \;
  \If{$\text{child} \neq \Nil$ \And $\text{child}.\field{parent} = \text{cap\_id}$ \And \Not $\text{child}.\field{revoked}$}{
    \textsc{Cap-Revoke}($\text{owner\_pid}$, $\text{child}.\field{cap\_id}$) \;
  }
}
\end{algorithm}

\begin{theorem}[Revocation Cascade Correctness]
\label{thm:revoke}
After \textsc{Cap-Revoke}($\text{pid}$, $c$), every capability $c'$ in the delegation
subtree rooted at $c$ satisfies $c'.\field{revoked} = \True$.
\end{theorem}
\begin{proof}
By structural induction on the delegation tree. The base case (leaf) is trivially
revoked. For an internal node, the algorithm recursively revokes all children whose
$\field{parent} = c.\field{cap\_id}$, covering the entire subtree.
\end{proof}

\section{Rights Model}

Grant operations enforce the \emph{monotonic attenuation} property:

\begin{property}[Monotonic Attenuation]
If capability $c_p$ (parent) has rights $R_p$ and grants capability $c_c$ (child) with
requested rights $R_c$, then $c_c.\field{rights} = R_c \cap R_p \subseteq R_p$.
A child can never possess more rights than its parent.
\end{property}

% =============================================================================
\chapter{Encrypted Database Engine}
\label{ch:database}

VaultOS enforces the principle ``all data is confidential'' by encrypting every record
at rest using per-table AES-128-CBC with HMAC-SHA256 integrity protection.
The Encrypt-then-MAC composition guarantees both confidentiality (IND-CPA) and
integrity (INT-CTXT), preventing both passive observation and active tampering of
stored records.

\section{Per-Table Key Derivation with Domain Separation}
\label{sec:key-derivation}

Each table requires two independent keys: an AES encryption key and an HMAC
authentication key.  Deriving both from the same HMAC output would constitute
\emph{key reuse}, so we apply domain separation by prepending distinct prefixes
to the HMAC input.

\begin{algorithm}[H]
\caption{\textsc{Derive-Table-Keys}($\text{table\_id}$): Derive AES and MAC keys with domain separation.}
\label{alg:derive-key}
\tcp{AES key derivation}
$\text{domain}_{\text{aes}} \leftarrow \texttt{"AES"} \,\|\, \text{LE32}(\text{table\_id})$ \tcp*{7-byte input}
$K_{\text{aes}} \leftarrow \textsc{HMAC-SHA256}(K_{\text{master}}, \text{domain}_{\text{aes}})[0..15]$ \;
$\textsc{AES-Init}(\text{aes\_ctx}[\text{table\_id}], K_{\text{aes}})$ \;
\BlankLine
\tcp{MAC key derivation}
$\text{domain}_{\text{mac}} \leftarrow \texttt{"MAC"} \,\|\, \text{LE32}(\text{table\_id})$ \tcp*{7-byte input}
$K_{\text{mac}} \leftarrow \textsc{HMAC-SHA256}(K_{\text{master}}, \text{domain}_{\text{mac}})$ \tcp*{full 32 bytes}
$\textsc{HMAC-Init}(\text{mac\_ctx}[\text{table\_id}], K_{\text{mac}})$ \;
\BlankLine
$\textsc{Memset-Zero}(K_{\text{aes}}, K_{\text{mac}})$ \tcp*{zeroize intermediates}
\end{algorithm}

The master key $K_{\text{master}}$ is a 256-bit value generated from RDRAND at
boot time. Since VaultOS has no persistent storage (yet), keys are ephemeral
and regenerated each boot cycle.

\section{Record Serialization}
\label{sec:record-serde}

Before encryption, a \texttt{record\_t} must be converted to a contiguous byte
buffer.  The wire format uses little-endian encoding:

\begin{center}
\begin{tabular}{lll}
\toprule
\textbf{Field} & \textbf{Bytes} & \textbf{Description} \\
\midrule
\texttt{row\_id}      & 8      & Primary key (uint64) \\
\texttt{table\_id}    & 4      & Table identifier (uint32) \\
\texttt{field\_count} & 4      & Number of fields \\
\midrule
\multicolumn{3}{l}{\itshape Per-field (repeated \texttt{field\_count} times):} \\
\texttt{type}  & 4              & Column type enum \\
\texttt{data}  & variable       & Type-dependent payload \\
\bottomrule
\end{tabular}
\end{center}

\noindent Field data sizes by type:

\begin{center}
\begin{tabular}{ll}
\toprule
\textbf{Type}        & \textbf{Data encoding} \\
\midrule
\texttt{COL\_U64}, \texttt{COL\_I64} & 8 bytes (little-endian) \\
\texttt{COL\_U32}    & 4 bytes \\
\texttt{COL\_U8}, \texttt{COL\_BOOL} & 1 byte \\
\texttt{COL\_STR}    & 2-byte length prefix $+$ \texttt{length} bytes \\
\texttt{COL\_BLOB}   & 4-byte length prefix $+$ \texttt{length} bytes \\
\bottomrule
\end{tabular}
\end{center}

Maximum serialized size is bounded by \texttt{MAX\_RECORD\_SIZE} (4096 bytes).
Both \textsc{Record-Serialize} and \textsc{Record-Deserialize} run in $\BigO(f)$
where $f$ is the field count, with bounds checking at every step.

\section{Encrypt-then-MAC Pipeline}
\label{sec:encrypt-pipeline}

Each record stored in the B-tree is wrapped in an \texttt{encrypted\_record\_t}:

\begin{lstlisting}[language=C,frame=single,basicstyle=\ttfamily\small]
typedef struct {
    uint8_t   iv[16];         /* Random IV for AES-CBC */
    uint8_t  *ciphertext;     /* Encrypted serialized record */
    uint32_t  ciphertext_len; /* Length (PKCS7-padded) */
    uint8_t   mac[32];        /* HMAC-SHA256(IV || ciphertext) */
    uint64_t  row_id;         /* Plaintext key for B-tree */
    uint32_t  table_id;       /* Table this belongs to */
} encrypted_record_t;
\end{lstlisting}

\begin{algorithm}[H]
\caption{\textsc{Record-Encrypt}($\text{rec}$, $\text{table\_id}$): Encrypt a record for storage.}
\label{alg:record-encrypt}
\tcp{Step 1: Serialize}
$\text{plain}[0..n{-}1] \leftarrow \textsc{Record-Serialize}(\text{rec})$ \;
\BlankLine
\tcp{Step 2: PKCS7 pad to AES block boundary}
$m \leftarrow \lceil n / 16 \rceil \times 16$ \;
$\textsc{PKCS7-Pad}(\text{plain}, n, m)$ \;
\BlankLine
\tcp{Step 3: Random IV}
$\text{iv}[0..15] \leftarrow \textsc{Random-Bytes}(16)$ \;
\BlankLine
\tcp{Step 4: AES-CBC encrypt (see \S\ref{sec:cbc})}
$\text{ct}[0..m{-}1] \leftarrow \textsc{AES-CBC-Encrypt}(\text{aes\_ctx}[\text{table\_id}], \text{iv}, \text{plain}, m)$ \;
\BlankLine
\tcp{Step 5: Authenticate (Encrypt-then-MAC)}
$\sigma \leftarrow \textsc{HMAC-Compute}(\text{mac\_ctx}[\text{table\_id}],\; \text{iv} \,\|\, \text{ct})$ \;
\BlankLine
\tcp{Step 6: Build encrypted record}
$\text{enc} \leftarrow \{\ \text{iv},\ \text{ct},\ m,\ \sigma,\ \text{rec.row\_id},\ \text{table\_id}\ \}$ \;
$\textsc{B-Tree-Insert}(\text{index}[\text{table\_id}],\ \text{rec.row\_id},\ \text{enc})$ \;
$\textsc{Memset-Zero}(\text{plain})$ \tcp*{zeroize plaintext buffer}
\end{algorithm}

\begin{figure}[htb]
\centering
\begin{tikzpicture}[
  box/.style={draw,thick,rounded corners=3pt,minimum width=1.9cm,
    minimum height=0.7cm,font=\sffamily\footnotesize,fill=blue!8},
  sec/.style={draw,thick,rounded corners=3pt,minimum width=1.9cm,
    minimum height=0.7cm,font=\sffamily\footnotesize,fill=green!12},
  arr/.style={->,>=Stealth,thick},
]
\node[box]                       (rec) {Record};
\node[box,right=0.4cm of rec]   (ser) {Serialize};
\node[box,right=0.4cm of ser]   (pad) {PKCS7 Pad};
\node[sec,right=0.4cm of pad]   (enc) {AES-CBC};
\node[sec,right=0.4cm of enc]   (mac) {HMAC};
\node[box,right=0.4cm of mac]   (sto) {B-Tree};
\draw[arr] (rec) -- (ser);
\draw[arr] (ser) -- (pad);
\draw[arr] (pad) -- (enc);
\draw[arr] (enc) -- (mac);
\draw[arr] (mac) -- (sto);
\node[above=0.15cm of enc,font=\sffamily\tiny\itshape] {random IV};
\end{tikzpicture}
\caption{Encrypt-then-MAC pipeline for record insertion. Green nodes indicate
cryptographic operations.}
\label{fig:encrypt-pipeline}
\end{figure}

\section{Verify-then-Decrypt Pipeline}
\label{sec:decrypt-pipeline}

On read, the MAC is verified \emph{before} any decryption occurs.  This is
essential: attempting to decrypt tampered ciphertext could produce chosen-plaintext
oracle attacks if error messages leak information.

\begin{algorithm}[H]
\caption{\textsc{Record-Decrypt}($\text{enc}$, $\text{table\_id}$): Verify MAC then decrypt.}
\label{alg:record-decrypt}
\tcp{Step 1: Recompute MAC}
$\sigma' \leftarrow \textsc{HMAC-Compute}(\text{mac\_ctx}[\text{table\_id}],\; \text{enc.iv} \,\|\, \text{enc.ct})$ \;
\BlankLine
\tcp{Step 2: Constant-time verification (see \S\ref{sec:ct-verify})}
\If{$\neg\,\textsc{HMAC-Verify}(\text{enc.mac},\; \sigma',\; 32)$}{
  \textbf{log} ``MAC verification failed for row enc.row\_id'' \;
  \Return \textsc{NULL} \tcp*{reject tampered record}
}
\BlankLine
\tcp{Step 3: AES-CBC decrypt}
$\text{padded}[0..m{-}1] \leftarrow \textsc{AES-CBC-Decrypt}(\text{aes\_ctx}[\text{table\_id}],\; \text{enc.iv},\; \text{enc.ct},\; m)$ \;
\BlankLine
\tcp{Step 4: Remove PKCS7 padding}
$n \leftarrow \textsc{PKCS7-Unpad}(\text{padded}, m)$ \;
\If{$n = 0$}{\Return \textsc{NULL} \tcp*{invalid padding}}
\BlankLine
\tcp{Step 5: Deserialize}
$\text{rec} \leftarrow \textsc{Record-Deserialize}(\text{padded}[0..n{-}1])$ \;
$\textsc{Memset-Zero}(\text{padded})$ \tcp*{zeroize decrypted buffer}
\Return $\text{rec}$ \;
\end{algorithm}

\begin{theorem}[Encrypt-then-MAC Security]
Let $\mathcal{E}$ be AES-128-CBC (IND-CPA secure) and $\mathcal{M}$ be
HMAC-SHA256 (SUF-CMA secure).  The Encrypt-then-MAC composition
$\Pi = (\mathcal{E}, \mathcal{M})$ achieves both IND-CPA confidentiality
and INT-CTXT ciphertext integrity, provided the encryption and MAC keys
are independent~\emph{(guaranteed by domain-separated derivation,
Algorithm~\ref{alg:derive-key})}.
\end{theorem}

\noindent The decrypt buffer is a static kernel variable, safe because VaultOS
is single-threaded.  All plaintext buffers are zeroed after use to limit the
window of exposure in physical memory.

\begin{remark}[Single-Threaded Precondition]
The use of a static decrypt buffer in Algorithm~\ref{alg:record-decrypt} relies
on the invariant that VaultOS never preempts a thread while a database operation
is in progress.  If future versions introduce preemptive multitasking or kernel
threads, the decrypt pipeline must be modified to use per-context buffers
(e.g., allocated on the caller's stack or from a per-CPU slab), or protected by
a spinlock.  Violating this precondition would allow a concurrent query to
overwrite a partially-filled plaintext buffer, leading to data corruption or
information leakage across security domains.
\end{remark}

\section{Constant-Time Audit Logging}
\label{sec:audit-ct}

The audit subsystem writes to \texttt{AuditTable} for every INSERT, DELETE,
and UPDATE operation.  Naive string handling leaks information about the
action and result strings through execution timing.  VaultOS pads all audit
strings to the fixed maximum length before serialization:

\begin{algorithm}[H]
\caption{\textsc{Audit-Log}($\text{pid}$, $\text{action}$, $\text{target\_id}$, $\text{result}$): Constant-time audit write.}
\label{alg:audit-log}
$\text{padded\_action}[0..\textsc{MaxStr}] \leftarrow \mathbf{0}^{\textsc{MaxStr}+1}$ \;
$\text{padded\_result}[0..\textsc{MaxStr}] \leftarrow \mathbf{0}^{\textsc{MaxStr}+1}$ \;
$\textsc{Memcpy}(\text{padded\_action},\; \text{action},\; \min(|\text{action}|, \textsc{MaxStr}))$ \;
$\textsc{Memcpy}(\text{padded\_result},\; \text{result},\; \min(|\text{result}|, \textsc{MaxStr}))$ \;
\BlankLine
$\text{rec} \leftarrow \textsc{New-Record}(\textsc{AuditTable})$ \;
$\text{rec.fields} \leftarrow [\text{row\_id},\; \text{timestamp},\; \text{pid},\;
  \text{padded\_action},\; \text{target\_id},\; \text{padded\_result}]$ \;
$\textsc{Record-Encrypt}(\text{rec},\; \textsc{TABLE\_ID\_AUDIT})$ \;
$\textsc{Memset-Zero}(\text{padded\_action},\; \text{padded\_result})$ \;
\end{algorithm}

By always writing $\textsc{MaxStr} = 255$ bytes regardless of actual string length,
the serialization, padding, encryption, and MAC computation all operate on identically
sized inputs, eliminating timing side-channels in the audit path.

\section{Query Execution Pipeline}
\label{sec:query-pipeline}

With Encrypt-then-MAC active, the query pipeline includes cryptographic operations
at both ends:

\begin{figure}[htb]
\centering
\begin{tikzpicture}[
  box/.style={draw,thick,rounded corners=3pt,minimum width=1.8cm,
    minimum height=0.7cm,font=\sffamily\footnotesize,fill=blue!8},
  sec/.style={draw,thick,rounded corners=3pt,minimum width=1.8cm,
    minimum height=0.7cm,font=\sffamily\footnotesize,fill=green!12},
  arr/.style={->,>=Stealth,thick},
  lab/.style={font=\sffamily\tiny\itshape,text=gray},
]
% SELECT path
\node[lab] at (-1.2, 0.45) {SELECT:};
\node[box]                       (tok) {Tokenize};
\node[box,right=0.35cm of tok]   (par) {Parse};
\node[box,right=0.35cm of par]   (exe) {B-Tree Scan};
\node[sec,right=0.35cm of exe]   (ver) {MAC Verify};
\node[sec,right=0.35cm of ver]   (dec) {Decrypt};
\node[box,right=0.35cm of dec]   (fil) {WHERE};
\node[box,right=0.35cm of fil]   (res) {Result};
\draw[arr] (tok) -- (par);
\draw[arr] (par) -- (exe);
\draw[arr] (exe) -- (ver);
\draw[arr] (ver) -- (dec);
\draw[arr] (dec) -- (fil);
\draw[arr] (fil) -- (res);

% INSERT path
\node[lab] at (-1.2, -1.15) {INSERT:};
\node[box,below=0.8cm of tok]    (tok2) {Tokenize};
\node[box,right=0.35cm of tok2]  (par2) {Parse};
\node[box,right=0.35cm of par2]  (ser)  {Serialize};
\node[sec,right=0.35cm of ser]   (enc)  {Encrypt};
\node[sec,right=0.35cm of enc]   (mac)  {MAC};
\node[box,right=0.35cm of mac]   (sto)  {B-Tree};
\node[box,right=0.35cm of sto]   (aud)  {Audit};
\draw[arr] (tok2) -- (par2);
\draw[arr] (par2) -- (ser);
\draw[arr] (ser) -- (enc);
\draw[arr] (enc) -- (mac);
\draw[arr] (mac) -- (sto);
\draw[arr] (sto) -- (aud);
\end{tikzpicture}
\caption{Query execution pipelines for SELECT and INSERT.  Green nodes indicate
cryptographic operations.  UPDATE combines both paths: decrypt$\to$modify$\to$re-encrypt
with a fresh IV.}
\label{fig:query-pipeline}
\end{figure}

\noindent For UPDATE queries, the pipeline decrypts matching records, applies field
modifications in plaintext, then re-encrypts with a fresh random IV and recomputed MAC.
This ensures that updating a record produces completely different ciphertext even if
the plaintext change is minimal (semantic security).

% =============================================================================
\chapter{Query Parser}
\label{ch:query}

\section{SQL Subset Grammar}

The parser accepts the following BNF grammar:

\begin{lstlisting}[language={},frame=none,numbers=none,
  basicstyle=\ttfamily\small,xleftmargin=0em]
<query>   ::= <select> | <insert> | <delete> | <update>
            | <show> | <describe> | <grant> | <revoke>
<select>  ::= SELECT <cols> FROM <ident> [WHERE <conds>]
<insert>  ::= INSERT INTO <ident> (<cols>) VALUES (<vals>)
<delete>  ::= DELETE FROM <ident> [WHERE <conds>]
<update>  ::= UPDATE <ident> SET <assigns> [WHERE <conds>]
<show>    ::= SHOW TABLES
<describe>::= DESCRIBE <ident>
<grant>   ::= GRANT <rights> ON <number> TO <number>
<revoke>  ::= REVOKE <number>
<conds>   ::= <cond> [AND <cond>]*
<cond>    ::= <ident> <op> <value>
<op>      ::= '=' | '!=' | '<' | '>' | '<=' | '>='
\end{lstlisting}

\begin{remark}[No DDL Support]
VaultOS does not support Data Definition Language statements
(\texttt{CREATE TABLE}, \texttt{DROP TABLE}, \texttt{ALTER TABLE}).
The six system tables are defined at compile time in
\texttt{db\_init\_system\_tables()} and cannot be created or destroyed at
runtime.  This is by design: the table schemas are part of the kernel's
security invariants, and allowing runtime schema modification would
undermine the static guarantees provided by the capability system.
\end{remark}

\section{Recursive-Descent Parser}

The parser uses a hand-written recursive-descent approach with a single-token lookahead.
Each non-terminal in the grammar maps to a function:

\begin{itemize}
\item \textsc{Parse-Select}(): handles \texttt{SELECT} queries
\item \textsc{Parse-Insert}(): handles \texttt{INSERT INTO} queries
\item \textsc{Parse-Where}(): parses \texttt{WHERE} clause conditions
\item \textsc{Next-Token}(): lexical scanner producing token + value pairs
\end{itemize}

\begin{theorem}[Parser Correctness]
The recursive-descent parser accepts exactly the language defined by the grammar above,
rejecting all other inputs with a syntax error containing the offending token position.
\end{theorem}

The parser runs in $\BigO(m)$ time where $m$ is the query string length, since each
character is examined at most once by the tokenizer, and each token is consumed exactly
once by the parser.

% =============================================================================
% PART VI: PROCESS MANAGEMENT
% =============================================================================
\part{Process Management}

% =============================================================================
\chapter{Processes and Scheduling}
\label{ch:scheduler}

\section{Process Control Block}

\begin{lstlisting}[caption={Process structure (simplified).}]
typedef struct process {
    uint64_t      pid;
    char          name[64];
    proc_state_t  state;
    context_t     context;      /* CPU: rsp, rip, cr3, regs */
    uint64_t      stack_base;   /* 64 KiB kernel stack */
    uint64_t      cap_root;     /* root capability ID */
    list_node_t   sched_node;   /* scheduler queue link */
} process_t;
\end{lstlisting}

\section{Process State Transitions}

\begin{figure}[htb]
\centering
\begin{tikzpicture}[
  state/.style={draw,thick,circle,minimum size=1.5cm,font=\sffamily\footnotesize},
  arr/.style={->,>=Stealth,thick},
  lab/.style={font=\footnotesize\itshape,above,sloped},
]
\node[state,fill=yellow!20]    (R) {Ready};
\node[state,fill=green!20,right=3cm of R] (U) {Running};
\node[state,fill=orange!20,below=2cm of U] (B) {Blocked};
\node[state,fill=red!15,left=3cm of B]  (T) {Terminated};

\draw[arr] (R) -- node[lab] {scheduled} (U);
\draw[arr] (U) -- node[lab] {preempted} (R);
\draw[arr] (U) -- node[lab,below,sloped] {I/O wait} (B);
\draw[arr] (B) -- node[lab,below,sloped] {I/O done} (R);
\draw[arr] (U) -- node[lab] {exit} (T);
\end{tikzpicture}
\caption{Process state transition diagram.}
\label{fig:proc-states}
\end{figure}

\section{Algorithm: \textsc{Round-Robin-Schedule}}

\begin{algorithm}[H]
\caption{\textsc{Schedule}(): Select the next process to run.}
\label{alg:schedule}
\If{ready\_queue is empty}{\Return}
$\text{next} \leftarrow$ dequeue from head of ready\_queue \;
\If{$\text{next} = \text{current\_process}$}{\Return}
\If{$\text{current\_process}.\field{state} = \const{Running}$}{
  $\text{current\_process}.\field{state} \leftarrow \const{Ready}$ \;
  enqueue current\_process at tail of ready\_queue \;
}
$\text{next}.\field{state} \leftarrow \const{Running}$ \;
set TSS.RSP0 to next's kernel stack top \;
$\text{old} \leftarrow \text{current\_process}$ \;
$\text{current\_process} \leftarrow \text{next}$ \;
\textsc{Context-Switch}($\&\text{old}.\field{context}$, $\&\text{next}.\field{context}$) \;
\end{algorithm}

\begin{property}[Fairness]
With a timeslice of 10 ticks (10\,ms) and $n$ ready processes, each process receives
at least $\lfloor 1000 / (10n) \rfloor$ scheduling quanta per second, ensuring bounded
response time of $10n$\,ms in the worst case.
\end{property}

\section{Context Switching}

The context switch saves and restores only callee-saved registers (per the System~V
AMD64 ABI): \reg{rbx}, \reg{rbp}, \reg{r12}--\reg{r15}, \reg{rsp}, \reg{rip},
and \reg{cr3} (page table base).

\begin{lstlisting}[caption={Context switch (x86-64 assembly, simplified).},
  language={[x86masm]Assembler}]
context_switch:
    ; Save old context (rdi = &old_ctx)
    mov [rdi+0x00], rbx
    mov [rdi+0x08], rbp
    mov [rdi+0x10], r12
    ; ... save r13-r15, rsp, rip, cr3 ...

    ; Load new context (rsi = &new_ctx)
    mov cr3, [rsi+0x48]     ; switch page tables
    mov rbx, [rsi+0x00]
    mov rbp, [rsi+0x08]
    mov rsp, [rsi+0x38]
    ; ... restore r12-r15 ...
    jmp [rsi+0x40]          ; resume at new rip
\end{lstlisting}

% =============================================================================
\chapter{Inter-Process Communication}
\label{ch:ipc}

\section{Message-Passing Model}

\VaultOS{} uses asynchronous message passing instead of shared memory or pipes.
Messages are stored in a circular buffer and recorded in the \texttt{MessageTable}
for auditing.

\section{Message Queue}

\begin{definition}[IPC Message]
An IPC message is a tuple:
\[
m = (\text{msg\_id},\; \text{src\_pid},\; \text{dst\_pid},\;
     \text{type},\; \text{payload}[0..511],\; \text{timestamp})
\]
with a maximum payload of 512 bytes.
\end{definition}

The message queue has capacity $C = 64$ messages.

\section{Algorithms: \textsc{IPC-Send} and \textsc{IPC-Recv}}

\begin{algorithm}[H]
\caption{\textsc{IPC-Send}($\text{src}$, $\text{dst}$, $\text{type}$, $\text{payload}$, $\text{len}$): Send a message.}
$\text{next\_head} \leftarrow (\text{head} + 1) \bmod C$ \;
\If{$\text{next\_head} = \text{tail}$}{\Return \const{Full}}
$\text{queue}[\text{head}] \leftarrow (++\text{msg\_id}, \text{src}, \text{dst}, \text{type}, \text{payload}, \text{now})$ \;
$\text{head} \leftarrow \text{next\_head}$ \;
\Return \const{Ok} \;
\end{algorithm}

\begin{algorithm}[H]
\caption{\textsc{IPC-Recv}($\text{pid}$): Receive first message addressed to $\text{pid}$.}
\For{$i \leftarrow \text{tail}$ \To $\text{head} - 1$ (modular)}{
  \If{$\text{queue}[i].\field{dst\_pid} = \text{pid}$}{
    $m \leftarrow \text{queue}[i]$ \;
    remove entry $i$ from queue (shift left) \;
    \Return $m$ \;
  }
}
\Return \Nil \tcp*{no messages}
\end{algorithm}

% =============================================================================
% PART VII: HARDWARE ABSTRACTION
% =============================================================================
\part{Hardware Abstraction}

% =============================================================================
\chapter{x86-64 Architecture Support}
\label{ch:x86}

\section{GDT and TSS Configuration}

The Global Descriptor Table contains 7 entries:

\begin{center}
\begin{tabular}{@{}clcc@{}}
\toprule
\textbf{Selector} & \textbf{Segment} & \textbf{DPL} & \textbf{Type} \\
\midrule
\hex{00} & Null       & -- & -- \\
\hex{08} & Kernel Code & 0 & 64-bit, exec, read \\
\hex{10} & Kernel Data & 0 & read/write \\
\hex{18} & User Data   & 3 & read/write \\
\hex{20} & User Code   & 3 & 64-bit, exec, read \\
\hex{28} & TSS (low)   & 0 & 64-bit TSS \\
\hex{30} & TSS (high)  & -- & upper 32 bits of TSS base \\
\bottomrule
\end{tabular}
\end{center}

\noindent
The TSS provides the \reg{RSP0} field used by the CPU when transitioning from Ring~3 to
Ring~0 on interrupts.

\section{Interrupt Handling}

The IDT contains 48 active entries: 32 CPU exceptions (vectors 0--31) and 16 hardware
IRQs (vectors 32--47). Each ISR stub follows the protocol:

\begin{enumerate}
\item Push dummy error code (if CPU did not push one).
\item Push vector number.
\item Push all 15 general-purpose registers.
\item Call \texttt{isr\_handler(interrupt\_frame\_t *frame)} in C.
\item Restore registers and execute \texttt{iretq}.
\end{enumerate}

\section{8259 PIC Initialization}

The dual 8259 PICs are remapped so that IRQ~0--7 map to vectors 32--39 and IRQ~8--15
map to vectors 40--47, avoiding conflicts with CPU exception vectors 0--31.

\section{SYSCALL/SYSRET Interface}

The SYSCALL mechanism uses three MSRs:
\begin{itemize}
\item \textbf{LSTAR} (\hex{C0000082}): kernel entry point address.
\item \textbf{STAR} (\hex{C0000081}): segment selectors (kernel CS/SS in bits 47:32, user CS/SS in bits 63:48).
\item \textbf{SFMASK} (\hex{C0000084}): flags to clear on SYSCALL (IF, TF).
\end{itemize}

The syscall calling convention passes the syscall number in \reg{rax} and arguments in
\reg{rdi}, \reg{rsi}, \reg{rdx}, \reg{r10}, \reg{r8}.

\section{CPUID Feature Detection}

\VaultOS{} queries \texttt{CPUID} leaf 1 to detect:
\begin{itemize}
\item \textbf{AES-NI}: ECX bit 25 --- enables hardware AES acceleration.
\item \textbf{SSE4.2}: ECX bit 20 --- available via \texttt{-march=x86-64-v2}.
\item \textbf{RDRAND}: ECX bit 30 --- hardware random number generation.
\end{itemize}

% =============================================================================
\chapter{Device Drivers}
\label{ch:drivers}

\section{Serial Port (COM1)}

The serial driver communicates at 115{,}200 baud via I/O port \hex{3F8}.
It provides \texttt{serial\_putchar()} and \texttt{serial\_write()} for debug output,
which is mirrored by \texttt{kprintf()} to both serial and framebuffer.

\section{GOP Framebuffer}

The UEFI Graphics Output Protocol (GOP) provides a linear framebuffer. The driver
renders text using an 8$\times$16 bitmap font, maintaining a cursor position and
supporting scroll via \texttt{memmove} of the framebuffer contents.

\begin{definition}[Text Grid]
For a framebuffer of $W \times H$ pixels with an $8 \times 16$ font, the text grid has
$\lfloor W/8 \rfloor$ columns and $\lfloor H/16 \rfloor$ rows. At $1024 \times 768$:
128 columns $\times$ 48 rows $= 6{,}144$ character cells.
\end{definition}

\section{PS/2 Keyboard}

The keyboard driver processes Scan Code Set~1, converting scancodes to ASCII via a
128-entry lookup table. It handles Shift and Caps Lock modifiers and buffers input in
a 256-byte ring buffer for consumption by \texttt{keyboard\_getchar()}.

\section{PS/2 Mouse}
\label{sec:mouse}

The mouse driver handles IRQ12 via the 8042 PS/2 controller (data port \hex{60},
command port \hex{64}). Initialization enables the auxiliary port, sets sample rate to
100~Hz, and enables data reporting. Each mouse event produces a 3-byte packet:

\begin{center}
\begin{tabular}{@{}cl@{}}
\toprule
\textbf{Byte} & \textbf{Contents} \\
\midrule
0 & Flags: buttons (bits 0--2), sign bits (4--5), overflow (6--7) \\
1 & $\Delta x$ (9-bit signed, sign in byte~0 bit~4) \\
2 & $\Delta y$ (9-bit signed, sign in byte~0 bit~5) \\
\bottomrule
\end{tabular}
\end{center}

\noindent
The driver accumulates bytes in a 3-byte state machine (triggered by the byte~0 alignment
bit~3) and pushes completed packets into a ring buffer. The GUI event loop calls
\texttt{mouse\_poll()} to dequeue packets and update cursor coordinates, clamped to
screen bounds.

\section{8$\times$16 Bitmap Font}

The font data is a compile-time constant: 256 glyphs $\times$ 16 bytes per glyph $=
4{,}096$ bytes. Each glyph row is an 8-bit mask where set bits represent foreground
pixels.

% =============================================================================
% PART IV: USER INTERFACE
% =============================================================================
\part{User Interface}

% -----------------------------------------------------------------------------
\chapter{VaultShell}
\label{ch:shell}

VaultShell is the primary user interface of \VaultOS{}: a text-mode shell that accepts
SQL queries and friendly commands, with syntax highlighting, tab completion, command
history, and a structured TUI layout.

% -----------------------------------------------------------------------------
\section{Text User Interface}
\label{sec:tui}

The shell screen is divided into four regions (Figure~\ref{fig:tui-layout}):

\begin{figure}[htb]
\centering
\begin{tikzpicture}[
  region/.style={draw,thick,minimum width=10cm,font=\sffamily\small},
]
\node[region,fill=blue!30,minimum height=0.6cm,text=white] (title)
  {Title Bar: ``VaultOS v0.1'' \quad\quad PID: 1 \quad State: Running};
\node[region,fill=gray!5,minimum height=4cm,below=0pt of title] (content)
  {Content Area (scrollable output + input prompt)};
\node[region,fill=blue!10,minimum height=0.6cm,below=0pt of content] (status)
  {Status Bar: Uptime \quad Heap \quad Tables \quad Processes};
\node[region,fill=gray!15,minimum height=0.6cm,below=0pt of status] (fkeys)
  {F1:Help \quad F2:Tables \quad F3:Status \quad F4:Quick \quad F5:Clear};
\end{tikzpicture}
\caption{VaultShell TUI layout. The status bar refreshes every 500\,ms.}
\label{fig:tui-layout}
\end{figure}

\noindent
The TUI layout is computed from the framebuffer dimensions: title bar occupies row~0,
F-key bar occupies the last row, status bar the row above it, and the content area fills
all remaining rows. The status bar updates asynchronously during the line editor's idle
loop.

% -----------------------------------------------------------------------------
\section{Line Editor}
\label{sec:line-editor}

The line editor provides in-place editing with visual feedback:

\begin{itemize}[leftmargin=2em]
  \item \textbf{Cursor movement}: Left/Right, Home/End.
  \item \textbf{Editing}: character insert at cursor, Backspace, Delete, Escape (clear).
  \item \textbf{History}: Up/Down arrows navigate a 32-entry ring buffer of previous commands.
        The current input is saved before entering history mode and restored when navigating
        past the most recent entry.
  \item \textbf{Syntax highlighting}: each character is colored in real time:
    \begin{itemize}
      \item SQL keywords and friendly verbs: highlight color (gold).
      \item Table aliases (\texttt{procs}, \texttt{caps}, etc.): cyan.
      \item String literals (single-quoted): green.
      \item Numbers: cyan. Operators (\texttt{=}, \texttt{<}, \texttt{>}, \texttt{*}): yellow.
    \end{itemize}
  \item \textbf{Block cursor}: the character at the cursor position is rendered with
        inverted foreground/background colors.
\end{itemize}

% -----------------------------------------------------------------------------
\section{Display Formatter}
\label{sec:display}

Query results are rendered as colored columnar tables. The formatter handles three cases:
\textsc{Show Tables} (table list with row counts), \textsc{Describe} (column schema with
types and flags), and general \textsc{Select} results (aligned columns with type-aware
formatting). Errors are displayed in red with the error message.

% -----------------------------------------------------------------------------
\section{Algorithm: Context-Aware Tab Completion}
\label{sec:completion}

Tab completion analyzes the tokens before the cursor to determine context.

\begin{algorithm}[H]
\caption{\textsc{Complete-Find}($\text{line}$, $\text{cursor\_pos}$): context-aware completion.}
\label{alg:complete}
$w \leftarrow$ extract word fragment before cursor\_pos \;
\If{$|w| = 0$}{\Return $\emptyset$}
$\text{tokens}[] \leftarrow$ tokenize line up to start of $w$ \;
\BlankLine
\tcp{Context: verb + table $\Rightarrow$ complete column names}
\If{$|\text{tokens}| \geq 2$ \And is\_friendly\_verb(tokens[0]) \And $w$ has no \texttt{=}}{
  $\text{schema} \leftarrow$ resolve\_table(tokens[1]) \;
  \If{schema $\neq$ \Nil}{
    \For{each column $c$ in schema}{
      \If{$c.\field{name}$ starts with $w$ (case-insensitive)}{
        add\_match($c.\field{name}$ + ``='') \;
      }
    }
    \If{matches $\neq \emptyset$}{\Return matches}
  }
}
\BlankLine
\tcp{Context: ``create'' $\Rightarrow$ complete object types}
\If{$|\text{tokens}| = 1$ \And tokens[0] = ``create''}{
  match $w$ against \{note, file, config, script, key\} \;
}
\tcp{Context: ``spawn'' $\Rightarrow$ complete program names}
\If{$|\text{tokens}| = 1$ \And tokens[0] = ``spawn''}{
  match $w$ against builtin programs $\cup$ \{``script:''\} \;
}
\BlankLine
\tcp{Context: verb alone $\Rightarrow$ complete table names + aliases}
\If{$|\text{tokens}| = 1$ \And is\_friendly\_verb(tokens[0])}{
  match $w$ against table names $\cup$ aliases \;
}
\BlankLine
\tcp{Default: match against all keywords, verbs, tables, aliases}
match $w$ against SQL keywords $\cup$ friendly verbs $\cup$ table names $\cup$ aliases \;
compute\_common\_prefix(matches) \;
\Return matches \;
\end{algorithm}

\noindent
When a single match is found, the remaining characters plus a trailing space are inserted.
When multiple matches exist, the longest common prefix is inserted and all candidates are
displayed below the input line in cyan on dark gray.

% =============================================================================
\chapter{Friendly Command Layer}
\label{ch:friendly}

The friendly command layer provides a natural-language-inspired interface that translates
user commands into SQL queries, eliminating the need to know SQL syntax for common
operations.

% -----------------------------------------------------------------------------
\section{Design: Commands as SQL Generators}
\label{sec:friendly-design}

\begin{definition}[Friendly Command]
A friendly command $F(v, t, \mathit{args})$ where $v$ is a verb, $t$ is a table reference,
and $\mathit{args}$ is a (possibly empty) sequence of key=value pairs, is a function that
produces a SQL query string $Q$ such that $\texttt{db\_execute}(Q, \mathit{pid})$ achieves
the intended semantics of~$F$.
\end{definition}

\noindent
This design preserves a critical invariant: \emph{the kernel database engine is never
modified}. The friendly layer is purely a shell-side string transformation, ensuring that
all security properties (capability checks, encryption, audit logging) remain intact.

% -----------------------------------------------------------------------------
\section{Command-to-SQL Translation}
\label{sec:friendly-commands}

Table~\ref{tab:friendly-cmds} lists the friendly verbs and their SQL translations.

\begin{table}[htb]
\centering
\caption{Friendly commands and their SQL translations.}
\label{tab:friendly-cmds}
\begin{tabular}{@{}llp{6cm}@{}}
\toprule
\textbf{Command} & \textbf{SQL Generated} & \textbf{Notes} \\
\midrule
\texttt{tables}              & \texttt{SHOW TABLES}              & \\
\texttt{show <table>}        & \texttt{SELECT * FROM <T>}        & Alias resolved \\
\texttt{info <table>}        & \texttt{DESCRIBE <T>}             & \\
\texttt{find <T> col=val}    & \texttt{SELECT * FROM <T> WHERE col = val} & Multiple conditions with AND \\
\texttt{count <T>}           & \texttt{SELECT * FROM <T>}        & Displays row count \\
\texttt{add <T> c=v c=v}     & \texttt{INSERT INTO <T> (...) VALUES (...)} & \\
\texttt{del <T> col=val}     & \texttt{DELETE FROM <T> WHERE col = val}    & \\
\texttt{set <T> c=v where k=v} & \texttt{UPDATE <T> SET c=v WHERE k=v}    & \\
\texttt{create <type> <name> [data]} & \texttt{INSERT INTO ObjectTable ...} & \\
\texttt{open <name>}         & \texttt{SELECT * FROM ObjectTable WHERE name='...'} & \\
\texttt{list [type]}         & \texttt{SELECT * FROM ObjectTable [WHERE type='...']} & \\
\texttt{rm <name>}           & \texttt{DELETE FROM ObjectTable WHERE name='...'} & \\
\texttt{ps}                  & \texttt{SELECT * FROM ProcessTable} & \\
\bottomrule
\end{tabular}
\end{table}

% -----------------------------------------------------------------------------
\section{Table Aliases}
\label{sec:aliases}

To reduce typing, the alias system maps short names to full table names:

\begin{center}
\begin{tabular}{@{}ll@{\qquad}ll@{\qquad}ll@{}}
\toprule
\textbf{Alias} & \textbf{Table} & \textbf{Alias} & \textbf{Table} & \textbf{Alias} & \textbf{Table} \\
\midrule
\texttt{procs}   & ProcessTable    & \texttt{caps}    & CapabilityTable & \texttt{objects} & ObjectTable \\
\texttt{msgs}    & MessageTable    & \texttt{audit}   & AuditTable      & \texttt{sys}     & SystemTable \\
\texttt{config}  & SystemTable     &                  &                 &                  & \\
\bottomrule
\end{tabular}
\end{center}

\noindent
Aliases are resolved before SQL generation by \texttt{friendly\_resolve\_alias()}, allowing
commands like \texttt{show procs} instead of \texttt{SELECT * FROM ProcessTable}. Aliases
are also syntax-highlighted in cyan in the line editor.

% -----------------------------------------------------------------------------
\section{Script Engine}
\label{sec:scripts}

Scripts are sequences of commands stored as rows in the ObjectTable. Each line is a
separate row with \texttt{type = 'script'} and the script name in the \texttt{name} field.
Lines are ordered by \texttt{obj\_id} (monotonically increasing from sequential inserts).

\begin{observation}[Script Line Ordering]
The \texttt{size} field in ObjectTable cannot be used for line ordering because the query
engine (line~479 of \texttt{query.c}) unconditionally overwrites it with the length of
\texttt{data}. Sequential inserts produce monotonically increasing \texttt{obj\_id} values
in the B-tree, and \textsc{B-Tree-Scan} returns keys in order, guaranteeing correct
line ordering without an explicit sequence number.
\end{observation}

\begin{algorithm}[H]
\caption{\textsc{Script-Save}($\text{name}$): interactive multi-line script recording.}
\label{alg:script-save}
\texttt{DELETE FROM ObjectTable WHERE name = name AND type = 'script'} \;
$\text{line\_num} \leftarrow 0$ \;
\While{$\text{line\_num} < \const{Max\_Script\_Lines}$}{
  print ``\texttt{\ \ N>\ }'' \;
  $\text{line} \leftarrow$ \textsc{Line-Read}() \;
  \If{line = ``end''}{break}
  \If{line is empty}{continue}
  \texttt{INSERT INTO ObjectTable (name, type, data) VALUES (name, 'script', line)} \;
  $\text{line\_num} \leftarrow \text{line\_num} + 1$ \;
}
\end{algorithm}

\begin{algorithm}[H]
\caption{\textsc{Script-Run}($\text{name}$): execute a stored script.}
\label{alg:script-run}
$\text{rows} \leftarrow$ \texttt{SELECT * FROM ObjectTable WHERE name = name AND type = 'script'} \;
\If{rows is empty}{\Error ``Script not found''}
\For{each row $r$ in rows (ordered by obj\_id)}{
  $\text{cmd} \leftarrow r.\field{data}$ \;
  \If{\textsc{Friendly-Translate}(cmd) succeeds}{
    $\text{sql} \leftarrow$ translated result \;
  }\Else{
    $\text{sql} \leftarrow$ cmd \tcp{pass through as raw SQL}
  }
  $\text{result} \leftarrow \texttt{db\_execute}(\text{sql}, \text{pid})$ \;
  display result \;
}
\end{algorithm}

% -----------------------------------------------------------------------------
\section{Process Management Commands}
\label{sec:proc-cmds}

Process management commands are built-in shell operations that invoke kernel process APIs
directly, since they require function pointer arguments that cannot be expressed as SQL.

\paragraph{spawn.}
The \texttt{spawn} command creates a new process from a registry of built-in programs or
from a stored script. For scripts, the entry point reads the script name from the
process name field (\texttt{process\_t.name}), avoiding global state and race conditions:

\begin{lstlisting}[caption={Script process entry point.}]
void script_process_entry(void) {
    process_t *self = process_get_current();
    const char *name = self->name + 7; /* skip "script:" */
    script_run(name);
    process_exit(self, 0);
    for (;;) hlt();
}
\end{lstlisting}

\paragraph{kill.}
Terminates a process by PID. The shell process is protected: attempting to kill the
shell's own PID is rejected with an error message.

\paragraph{msg / inbox.}
\texttt{msg <pid> <text>} inserts a row into MessageTable with the destination PID and
payload. \texttt{inbox} queries MessageTable for messages addressed to the current process.

\paragraph{monitor.}
The \texttt{monitor} built-in program is a background process that runs in an infinite
loop, inserting a system stats record (uptime, heap usage) into AuditTable every 5
seconds. It demonstrates preemptive multitasking and database access from concurrent
processes.

% -----------------------------------------------------------------------------
\section{Algorithm: Friendly-Translate}
\label{sec:friendly-algo}

\begin{algorithm}[H]
\caption{\textsc{Friendly-Translate}($\text{input}$, $\text{sql\_buf}$): convert a friendly command to SQL.}
\label{alg:friendly}
$\text{verb} \leftarrow$ first token of input \;
$\text{rest} \leftarrow$ remaining tokens \;
\BlankLine
\uIf{verb = ``tables''}{
  $\text{sql\_buf} \leftarrow$ ``\texttt{SHOW TABLES}'' \;
}\uElseIf{verb = ``show''}{
  $t \leftarrow$ resolve\_alias(next token) \;
  $\text{sql\_buf} \leftarrow$ ``\texttt{SELECT * FROM } $t$'' \;
}\uElseIf{verb = ``find''}{
  $t \leftarrow$ resolve\_alias(next token) \;
  parse key=value pairs from rest into WHERE clauses with AND \;
  $\text{sql\_buf} \leftarrow$ ``\texttt{SELECT * FROM } $t$ \texttt{ WHERE ...}'' \;
}\uElseIf{verb = ``add''}{
  $t \leftarrow$ resolve\_alias(next token) \;
  parse key=value pairs into column and value lists \;
  $\text{sql\_buf} \leftarrow$ ``\texttt{INSERT INTO } $t$ \texttt{ (...) VALUES (...)}'' \;
}\uElseIf{verb $\in$ \{``create'', ``open'', ``list'', ``rm'', ``ps'', \ldots\}}{
  generate appropriate SQL for ObjectTable or ProcessTable \;
}\Else{
  \Return \False \tcp{not a friendly command}
}
\Return \True \;
\end{algorithm}

% =============================================================================
\chapter{Graphical Desktop}
\label{ch:gui}

\VaultOS{} includes a graphical desktop environment built on top of the GOP framebuffer.
The GUI provides windowed applications for database queries, table browsing, process
management, and system monitoring. The user can switch between TUI and GUI modes with
the \texttt{gui} command.

% -----------------------------------------------------------------------------
\section{Graphics Subsystem}
\label{sec:graphics}

The graphics layer implements double buffering over the GOP linear framebuffer:

\begin{itemize}[leftmargin=2em]
  \item \textbf{Back buffer}: a heap-allocated buffer of $W \times H \times 4$ bytes
        (32-bit ARGB). All drawing operations target this buffer.
  \item \textbf{Flip}: \texttt{gfx\_flip()} copies the back buffer to the framebuffer
        in a single \texttt{memcpy}. Partial flips (\texttt{gfx\_flip\_rect}) copy only
        the dirty region to minimize bandwidth.
\end{itemize}

\noindent
Primitive operations include \texttt{gfx\_fill\_rect()}, \texttt{gfx\_draw\_rect()} (outline),
\texttt{gfx\_draw\_hline()}, \texttt{gfx\_draw\_text()} (using the 8$\times$16 bitmap font),
and pixel-level access via \texttt{gfx\_putpixel()}.

\begin{definition}[Double Buffer Memory]
For a $1024 \times 768$ display at 32\,bpp, each buffer requires $1024 \times 768 \times 4
= 3{,}145{,}728$ bytes ($\approx 3$\,MiB). The heap is expanded by 15\,MiB upon GUI launch
to accommodate the back buffer, window canvases, and widget data.
\end{definition}

\noindent
The back buffer is allocated via \texttt{kmalloc()} from the kernel heap
(Chapter~\ref{ch:heap}), not from a dedicated video memory pool.  For the
default $1024 \times 768$ resolution this consumes $\approx 3$\,MiB of the
15\,MiB GUI heap expansion.  The remaining $\approx 12$\,MiB is available for
window canvases, widget trees, and compositor state.

% -----------------------------------------------------------------------------
\section{Event System}
\label{sec:events}

The event system translates hardware inputs into a unified event queue:

\begin{center}
\begin{tabular}{@{}ll@{}}
\toprule
\textbf{Event Type} & \textbf{Source} \\
\midrule
\texttt{EVT\_MOUSE\_DOWN/UP/MOVE} & PS/2 mouse packets \\
\texttt{EVT\_KEY\_PRESS/RELEASE}  & PS/2 keyboard scancodes \\
\texttt{EVT\_CLOSE}               & Window close button click \\
\bottomrule
\end{tabular}
\end{center}

\noindent
\texttt{event\_pump()} polls the mouse and keyboard drivers non-blockingly and enqueues
events into a fixed-size ring buffer. The main loop calls \texttt{event\_poll()} to dequeue
events for dispatch.

% -----------------------------------------------------------------------------
\section{Window Manager}
\label{sec:wm}

The window manager maintains an ordered list of windows (up to 8 simultaneous windows).
Each window has:

\begin{lstlisting}[caption={Window structure (simplified).}]
typedef struct window {
    uint32_t  id;
    char      title[64];
    int16_t   x, y;               /* screen position */
    uint16_t  w, h;               /* total size */
    uint16_t  client_w, client_h; /* client area */
    uint32_t *canvas;             /* pixel buffer */
    bool      focused, dragging;
    widget_t *widgets;            /* widget linked list */
    void (*on_event)(window_t *, gui_event_t *);
    void (*on_paint)(window_t *);
} window_t;
\end{lstlisting}

\noindent
Window operations include:

\begin{itemize}[leftmargin=2em]
  \item \textbf{Focus}: clicking a window brings it to front (z-order head).
  \item \textbf{Drag}: clicking the title bar and moving the mouse repositions the window.
  \item \textbf{Close}: clicking the ``X'' button sends \texttt{EVT\_CLOSE} to the window's
        event handler, which calls \texttt{wm\_destroy\_window()}.
  \item \textbf{Decorations}: the window manager draws a title bar (20\,px), 1\,px border,
        and close button. The client area is below the title bar.
\end{itemize}

% -----------------------------------------------------------------------------
\section{Compositor}
\label{sec:compositor}

The compositor renders the desktop each frame:

\begin{algorithm}[H]
\caption{\textsc{Compositor-Render}(): render one frame of the desktop.}
\label{alg:compositor}
clear back buffer to desktop background color \;
\For{each window $w$ in bottom-to-top z-order}{
  draw $w$'s border and title bar decorations \;
  call $w$.\texttt{on\_paint}() to render widgets into $w$'s canvas \;
  blit $w$'s canvas to back buffer at $(w.x, w.y)$ \;
}
draw mouse cursor (XOR blending for visibility) \;
\texttt{gfx\_flip\_rect}(dirty region) \;
\end{algorithm}

\noindent
The XOR-blended cursor ensures visibility against any background color. The compositor
runs in the shell's main loop at approximately 60 iterations per second (limited by
\texttt{hlt()} idle waits).

% -----------------------------------------------------------------------------
\section{Widget Toolkit}
\label{sec:widgets}

Four widget types are provided:

\begin{center}
\begin{tabular}{@{}lp{8cm}@{}}
\toprule
\textbf{Widget} & \textbf{Behavior} \\
\midrule
\texttt{Button}   & Click callback, hover highlight, text label \\
\texttt{Label}    & Static text with configurable foreground/background \\
\texttt{TextBox}  & Single-line text input with cursor, character insert/delete \\
\texttt{ListView} & Scrollable item list with selection highlight, click-to-select,
                    optional \texttt{on\_select} callback \\
\bottomrule
\end{tabular}
\end{center}

\noindent
Widgets are stored as a linked list per window. \texttt{widget\_dispatch()} routes mouse
and keyboard events to the appropriate widget based on hit testing. \texttt{widget\_draw\_all()}
renders all widgets in a window's canvas.

% -----------------------------------------------------------------------------
\section{Desktop Applications}
\label{sec:desktop-apps}

The GUI provides four applications accessible from the taskbar menu:

\paragraph{Query Console.}
A SQL REPL with a textbox for query input, ``Execute'' and ``Template'' buttons, and a
listview for results. Templates cycle through common queries (\texttt{SHOW TABLES},
\texttt{SELECT * FROM SystemTable}, etc.).

\paragraph{Table Browser.}
A two-pane interface: the left pane lists all system tables (with Refresh), and the right
pane shows either the table schema (click a table name) or query results (``View All''
button). A search box supports \texttt{column=value} filtering.

\paragraph{Process Manager.}
Lists all processes from ProcessTable with Refresh, ``Spawn Monitor'' (creates a background
monitoring process), and ``Kill'' buttons. The Kill button protects the shell process
from self-termination.

\paragraph{System Status.}
Displays live system metrics: uptime, heap usage (used/free), table count, and CPU
architecture. Labels update on each paint cycle.

\paragraph{Taskbar.}
The taskbar occupies the bottom 28 pixels: a ``VaultOS'' menu button on the left, window
buttons in the center, and a status line (uptime + heap) on the right. Clicking a
window button brings that window to the front.

% =============================================================================
% APPENDICES
% =============================================================================
\appendix

\chapter{Virtual Address Space Map}
\label{app:vas}

\begin{center}
\begin{tabular}{@{}ll@{}}
\toprule
\textbf{Virtual Address Range} & \textbf{Purpose} \\
\midrule
\texttt{0x0000000000200000} -- \texttt{0x00007FFFFFFFE000} & User process code and stack \\
\texttt{0xFFFFFFFF80000000} -- \texttt{0xFFFFFFFF81FFFFFF} & Kernel code and data (2\,MiB) \\
\texttt{0xFFFFFFFF82000000} -- \texttt{0xFFFFFFFF91FFFFFF} & Kernel heap (256\,MiB) \\
\texttt{0xFFFFFFFF92000000} -- \texttt{0xFFFFFFFFBFFFFFFF} & Physical memory direct map \\
\texttt{0xFFFFFFFFC0000000} -- \texttt{0xFFFFFFFFCFFFFFFF} & Framebuffer \\
\bottomrule
\end{tabular}
\end{center}

\chapter{Syscall Number Table}
\label{app:syscalls}

\begin{center}
\begin{tabular}{@{}cll@{}}
\toprule
\textbf{Number} & \textbf{Name} & \textbf{Description} \\
\midrule
0 & \texttt{SYS\_DB\_QUERY} & Execute database query \\
1 & \texttt{SYS\_DB\_INSERT} & Insert record \\
2 & \texttt{SYS\_DB\_DELETE} & Delete record \\
3 & \texttt{SYS\_DB\_UPDATE} & Update record \\
10 & \texttt{SYS\_CAP\_GRANT} & Grant capability \\
11 & \texttt{SYS\_CAP\_REVOKE} & Revoke capability \\
12 & \texttt{SYS\_CAP\_DELEGATE} & Delegate capability \\
13 & \texttt{SYS\_CAP\_LIST} & List capabilities \\
20 & \texttt{SYS\_PROC\_CREATE} & Create process \\
21 & \texttt{SYS\_PROC\_EXIT} & Terminate process \\
22 & \texttt{SYS\_PROC\_INFO} & Query process info \\
30 & \texttt{SYS\_IPC\_SEND} & Send IPC message \\
31 & \texttt{SYS\_IPC\_RECV} & Receive IPC message \\
40 & \texttt{SYS\_IO\_READ} & Read I/O \\
41 & \texttt{SYS\_IO\_WRITE} & Write I/O \\
50 & \texttt{SYS\_INFO} & System information \\
\bottomrule
\end{tabular}
\end{center}

\chapter{Error Code Reference}
\label{app:errors}

\begin{center}
\begin{tabular}{@{}cll@{}}
\toprule
\textbf{Code} & \textbf{Name} & \textbf{Description} \\
\midrule
0   & \texttt{VOS\_OK}              & Success \\
$-1$ & \texttt{VOS\_ERR\_GENERIC}   & Generic error \\
$-2$ & \texttt{VOS\_ERR\_NOMEM}     & Out of memory \\
$-3$ & \texttt{VOS\_ERR\_INVAL}     & Invalid argument \\
$-4$ & \texttt{VOS\_ERR\_NOTFOUND}  & Record not found \\
$-5$ & \texttt{VOS\_ERR\_PERM}      & Permission denied \\
$-6$ & \texttt{VOS\_ERR\_EXISTS}    & Already exists \\
$-7$ & \texttt{VOS\_ERR\_FULL}      & Table/resource full \\
$-8$ & \texttt{VOS\_ERR\_SYNTAX}    & Query syntax error \\
$-9$ & \texttt{VOS\_ERR\_CAP\_INVALID} & HMAC verification failed \\
$-10$ & \texttt{VOS\_ERR\_CAP\_EXPIRED} & Capability expired \\
$-11$ & \texttt{VOS\_ERR\_CAP\_REVOKED} & Capability revoked \\
$-12$ & \texttt{VOS\_ERR\_TXN\_ABORT}  & Transaction aborted \\
\bottomrule
\end{tabular}
\end{center}

\chapter{Performance Benchmarks}
\label{app:benchmarks}

Measured on QEMU with 100 iterations per benchmark:

\begin{center}
\begin{tabular}{@{}lrrr@{}}
\toprule
\textbf{Benchmark} & \textbf{Haswell (AES-NI)} & \textbf{Nehalem (SW)} & \textbf{Speedup} \\
\midrule
AES-CBC encrypt 1\,KiB  & 29{,}388 cyc/op  & 217{,}929 cyc/op  & $7.4\times$ \\
AES-CBC decrypt 1\,KiB  & 39{,}023 cyc/op  & 284{,}357 cyc/op  & $7.3\times$ \\
SHA-256 1\,KiB           & 74{,}381 cyc/op  & 67{,}679 cyc/op   & $1.0\times$ \\
HMAC-SHA256 40\,B        & 21{,}109 cyc/op  & 21{,}201 cyc/op   & $1.0\times$ \\
\textsc{Cap-Check}       & 2{,}560 cyc/op   & 2{,}466 cyc/op    & $1.0\times$ \\
\bottomrule
\end{tabular}
\end{center}

\noindent
The AES-NI hardware acceleration provides a $7.3$--$7.4\times$ speedup for database
record encryption and decryption. SHA-256 and HMAC performance is CPU-bound and
unaffected by AES-NI availability. Capability validation (\textsc{Cap-Check}) benefits
primarily from the validation cache, not hardware acceleration.

\chapter{VaultShell Command Reference}
\label{app:commands}

\noindent
\begin{small}
\begin{tabular}{@{}llp{4.5cm}@{}}
\toprule
\textbf{Category} & \textbf{Command} & \textbf{Description} \\
\midrule
\multicolumn{3}{@{}l}{\itshape Query Commands} \\
& \texttt{tables}              & List all system tables \\
& \texttt{show <table>}        & Display all rows \\
& \texttt{info <table>}        & Show table schema \\
& \texttt{find <T> col=val}    & Search rows \\
& \texttt{count <table>}       & Count rows \\
\midrule
\multicolumn{3}{@{}l}{\itshape Mutation Commands} \\
& \texttt{add <T> col=val ...} & Insert a new row \\
& \texttt{del <T> col=val}     & Delete matching rows \\
& \texttt{set <T> c=v where k=v} & Update matching rows \\
\midrule
\multicolumn{3}{@{}l}{\itshape Object Commands} \\
& \texttt{create <type> <name>}& Create an object \\
& \texttt{open <name>}         & View an object \\
& \texttt{list [type]}         & List objects \\
& \texttt{rm <name>}           & Delete an object \\
& \texttt{cat <name>}          & Display contents \\
\midrule
\multicolumn{3}{@{}l}{\itshape Script Commands} \\
& \texttt{save <name>}         & Record a script \\
& \texttt{run <name>}          & Execute a script \\
& \texttt{scripts}             & List all scripts \\
\midrule
\multicolumn{3}{@{}l}{\itshape Process Commands} \\
& \texttt{ps}                  & List processes \\
& \texttt{spawn <program>}     & Launch a program \\
& \texttt{spawn script:<name>} & Run script as process \\
& \texttt{kill <pid>}          & Terminate a process \\
& \texttt{msg <pid> <text>}    & Send a message \\
& \texttt{inbox}               & View messages \\
\midrule
\multicolumn{3}{@{}l}{\itshape System \& SQL Commands} \\
& \texttt{help / status / clear} & Built-in utilities \\
& \texttt{gui}                 & Launch GUI desktop \\
& \texttt{SELECT / INSERT / ...} & Raw SQL pass-through \\
& \texttt{GRANT / REVOKE}      & Capability management \\
\midrule
\multicolumn{3}{@{}l}{\itshape F-Key Shortcuts} \\
& \texttt{F1--F5}             & Help, Tables, Status, Quick, Clear \\
\bottomrule
\end{tabular}
\end{small}

\end{document}
